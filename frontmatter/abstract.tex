The western monarch butterfly (\textit{Danaus plexippus}) population has experienced severe declines, with recent counts representing over 99\% reduction from 1980s abundance. Conservation of overwintering habitat along California's coast has emerged as a management priority. For over three decades, habitat management has been guided by the hypothesis that wind speeds exceeding 2 m/s disrupt butterfly clusters, forcing monarchs to abandon roosts. This threshold has shaped restoration efforts and management guidelines despite never undergoing direct empirical testing.

This study provides the first experimental test of wind effects on overwintering monarch clusters. We deployed remote monitoring equipment at two sites on Vandenberg Space Force Base during the 2023-2024 overwintering season, collecting 1,894 paired observations at 30-minute intervals across 78 days. Time-lapse photography captured butterfly abundance while co-located anemometers measured wind conditions at roost height.

Linear regression analysis revealed no relationship between wind speed and cluster size changes at either 30-minute intervals ($R^2$ = 0.002, p = 0.073) or day-to-day timescales ($R^2$ = 0.013, p = 0.252). Every observation period in our day-to-day analysis experienced maximum wind speeds exceeding the proposed 2 m/s threshold, yet clusters persisted throughout the study. Model selection using generalized additive mixed models identified temperature, time of day, and an interaction between wind and solar exposure as significant predictors of cluster dynamics, while wind alone showed no independent effect.

The wind-sunlight interaction revealed complex conditional relationships. When butterflies remained in shade, wind speed had no effect on cluster dynamics across the observed range (0 to 12.4 m/s). Effects emerged only when butterflies occupied sunlit positions, where moderate wind combined with sun exposure sometimes increased cluster sizes. These patterns, combined with strong effects of temperature and diurnal activity cycles, suggest that thermoregulation may better explain monarch clustering behavior than wind disruption.

Our findings challenge fundamental assumptions underlying monarch habitat management. The absence of wind disruption despite frequent threshold exceedances indicates that suitable habitat may be more extensive than currently recognized. Conservation efforts should reconsider management practices based on rigid wind speed thresholds and instead focus on understanding the thermal and light conditions that monarchs select when choosing roost sites. As western monarch populations face potential extinction, evidence-based understanding of overwintering ecology becomes essential for effective conservation of this remarkable migration phenomenon.