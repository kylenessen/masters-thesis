The western monarch butterfly (\textit{Danaus plexippus}) population has experienced severe declines, with recent counts representing over 99\% reduction from 1980s abundance. Conservation of overwintering habitat along California's coast has become a management priority. For over three decades, habitat management has followed the Disruptive Wind Hypothesis, positing that wind speeds exceeding 2 m/s disrupt butterfly clusters and cause monarchs to abandon roosts. This threshold has influenced restoration practices and management guidelines despite no direct empirical validation.

This study provides the first experimental test of wind effects on overwintering monarch clusters. We used cameras and wind sensors at two sites on Vandenberg Space Force Base during the 2023-2024 overwintering season, collecting 1,894 paired observations at 30-minute intervals across 78 days. Time-lapse photography documented butterfly abundance while wind sensors at roost height recorded wind conditions.

Statistical analysis found no relationship between wind speed alone and changes in cluster size at either 30-minute or daily timescales. Every day with butterflies in our study experienced wind speeds exceeding the proposed 2 m/s threshold, yet clusters persisted throughout the monitoring period. Advanced statistical models identified temperature, time of day, and an interaction between wind and sunlight as factors influencing cluster behavior. Wind alone showed no independent effect on cluster dynamics.

The interaction between wind and sunlight revealed unexpected patterns. When butterflies remained in shade, wind speed had no effect on cluster behavior across the entire observed range (0 to 12.4 m/s). Wind effects appeared only when butterflies occupied sunlit positions, where the combination of moderate wind and sun exposure sometimes corresponded with larger cluster sizes. These patterns, along with temperature effects and daily activity cycles, suggest that temperature regulation may drive monarch clustering behavior more than wind avoidance.

Our findings challenge fundamental assumptions in monarch habitat management. The absence of wind disruption despite consistent winds above the accepted threshold indicates that suitable habitat may be more extensive than currently recognized. Conservation efforts should reconsider practices based on wind speed thresholds. Instead, management should focus on understanding the thermal and light conditions that monarchs may select when choosing roost sites. As western monarch populations face severe declines, evidence-based understanding of overwintering ecology becomes essential for effective conservation.