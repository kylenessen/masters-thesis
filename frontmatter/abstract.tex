The western monarch butterfly (\textit{Danaus plexippus}) population has declined by over 99\% since the 1980s, making conservation of overwintering habitat along California's coast a management priority. For three decades, habitat management has followed the Disruptive Wind Hypothesis, which posits that wind speeds exceeding 2 m/s disrupt butterfly clusters and cause roost abandonment. This threshold has had substantial influence on restoration practices, despite lacking direct empirical validation.

This study provides the first experimental test of wind effects on overwintering monarch clusters. We deployed cameras and wind sensors at two sites on Vandenberg Space Force Base during the 2023-2024 overwintering season, collecting 1,894 paired observations at 30-minute intervals across 78 days. Time-lapse photography documented butterfly abundance while wind sensors at roost height recorded wind conditions.

Linear regression analysis revealed no significant relationship between wind speed and changes in cluster size at either 30-minute or daily timescales ($R^2$ = 0.002, p = 0.073 and $R^2$ = 0.013, p = 0.252, respectively). Every day with butterflies present experienced wind speeds exceeding the 2 m/s threshold, yet clusters persisted throughout the monitoring period. Generalized additive mixed models identified temperature, time of day, and an interaction between wind and sunlight as factors influencing cluster behavior (p $<$ 0.001), but wind alone showed no independent effect. When butterflies remained in shade, wind speed had no effect on cluster behavior across the entire observed range (0 to 12.4 m/s). Effects appeared only when butterflies occupied sunlit positions, suggesting that thermoregulation may be a primary driver of clustering behavior.

Our findings challenge fundamental assumptions in monarch habitat management. The absence of wind disruption indicates that monarchs may be more resilient to environmental variation than previously believed, with suitable habitat potentially more extensive than currently recognized. Conservation efforts should continue promoting healthy groves while reconsidering strict wind-based management requirements. Future research should investigate canopy structure and resulting light patterns that monarchs may select when choosing roost sites. As western monarch populations continue to face historic lows, evidence-based understanding of overwintering ecology becomes essential for effective conservation.