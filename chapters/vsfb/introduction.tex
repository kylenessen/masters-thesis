\usepackage{hyperref}
\usepackage{longtable}

\chapter{Introduction}
\label{ch:introduction}

The distribution and survival of invertebrate species are governed by a complex interplay of biotic and abiotic factors. Biotic interactions fundamentally shape invertebrate communities through predation pressure that drives cryptic coloration and refuge-seeking behaviors \citep{blois-heulinDirectIndirectEffects1990; holomuzkiBioticInteractionsFreshwater2010}, parasitism that alters host movement patterns and microhabitat selection \citep{jollesSchistocephalusParasiteInfection2020; laffertyComparingMechanismsHost2013; poulinMetaanalysisParasiteinducedBehavioural1994}, and competition for resources that influences spatial partitioning among species \citep{wertheimSpeciesDiversityMycophagous2000}. Mutualistic relationships, such as ant-aphid associations, further constrain partners to shared microhabitats \citep{wayMutualismAntsHoneydewProducing1963; yaoCostsConstraintsAphidant2014}.
Yet all these biological interactions operate within constraints imposed by the physical environment. Species must first survive the abiotic conditions of their habitat before biotic factors can shape community structure. Thus, while biotic interactions refine distributional patterns, abiotic conditions often determine the fundamental boundaries of where invertebrates can persist.

Abiotic factors fundamentally influence invertebrate distribution and behavior. Temperature regulates metabolic rates and activity periods, as seen in desert beetles that restrict foraging to narrow thermal windows. Humidity determines desiccation risk, particularly for soil-dwelling arthropods and other small-bodied invertebrates. Solar radiation enables activity in ectotherms while simultaneously imposing thermal stress, forcing trade-offs between energy gain and overheating. Precipitation drives resource pulses and population dynamics, especially in arid-adapted species. Time of day itself influences activity independent of environmental conditions, with many invertebrates exhibiting endogenous rhythms that persist even under constant laboratory conditions.

Wind influences invertebrate behavior through multiple mechanisms. At low velocities, wind can facilitate dispersal and migration, as demonstrated by spider ballooning behavior where individuals actively seek wind currents for long-distance transport \citep{bonteHeritabilitySpiderBallooning2007}. However, as wind speed increases, it transitions from facilitator to stressor. \citet{bellSearchAnemotacticOrientation1979} demonstrated that cockroaches alter their movement patterns at wind speeds as low as 0.03 m/s, shifting from looping search patterns to directed movement. Similarly, mosquito flight activity decreases by 50\% at wind speeds of just 0.5 m/s and by 75\% at 1.0 m/s, despite their typical flight speeds of approximately 1 m/s \citep{bidlingmayerEffectWindVelocity1995}.

These behavioral modifications are associated with energetic costs of maintaining position and orientation in moving air. \citet{leonardExposureWindAlters2016} found that caterpillar larvae exposed to moderate winds of 3 m/s in laboratory settings exhibited dramatic behavioral changes, including increased dropping from plants and movement to sheltered microsites on petioles, branch junctions, and the leeward sides of leaves. Wind also interferes with sensory systems: vibrational communication in homopterans becomes impossible when wind-induced plant vibrations exceed insect-generated signals, forcing these insects to restrict signaling to calm periods between gusts \citep{tishechkinVibrationalBackgroundNoise2013}. Some species have evolved to exploit wind patterns, such as neotropical katydids that concentrate their calling during predictable low-wind periods at night to avoid acoustic interference \citep{velillaGoneWindSignal2020}.

The diversity of wind responses across invertebrate taxa highlights the complexity of this abiotic factor. While some species actively exploit wind for dispersal, others dramatically alter behavior at relatively low wind speeds. The specific velocities triggering behavioral changes vary widely among species—from cockroaches responding at 0.03 m/s to katydids adapting to predictable wind patterns. This variation suggests that wind effects may be highly species-specific, shaped by body size, life history, and ecological context. Understanding how wind influences particular species, especially those of conservation concern, requires detailed empirical investigation rather than broad generalizations.

The monarch butterfly (\textit{Danaus plexippus}) presents an exceptional system for studying how abiotic factors influence invertebrate ecology across multiple scales. This iconic species has experienced severe population declines, with western monarchs declining by over 95\% from historical levels of 4.5 million in the 1980s to fewer than 30,000 individuals in recent counts \citep{schultzCitizenScienceMonitoring2017a, peltonWesternMonarchPopulation2019}. The severity of this decline has prompted consideration for listing under the Endangered Species Act, elevating the urgency of understanding factors limiting monarch populations \citep{croneWhyAreMonarch2019}.

Monarchs exhibit a complex life history characterized by distinct seasonal phases that expose them to varying environmental challenges. During spring and summer, successive generations disperse across North America, reproducing on milkweed plants and exploiting favorable conditions for population growth. As autumn approaches, a migratory generation emerges that foregoes reproduction and instead embarks on a remarkable journey to overwintering sites. Eastern monarchs migrate to high-altitude oyamel fir forests in central Mexico, while western monarchs converge on coastal groves in California, primarily between Mendocino County and northern Baja California. This overwintering phase represents a critical bottleneck in the monarch life cycle, where millions of butterflies must survive 3-5 months on fixed lipid reserves while avoiding predation, disease, and environmental stressors.

The selection of overwintering habitat by monarchs reflects a suite of environmental requirements that must be met simultaneously for survival. At the landscape scale, \citet{fisherClimaticNicheModel2018} used climatic niche modeling to demonstrate that suitable overwintering sites in California occur within a narrow coastal band characterized by mild temperatures, moderate humidity, and protection from extreme weather events. These macro-climatic conditions create the environmental envelope within which monarchs can successfully overwinter, but habitat selection operates at multiple nested scales.

Within suitable climatic zones, monarchs exhibit strong preferences for specific grove characteristics. The microclimate hypothesis, first articulated by \citet{leongMicroenvironmentalFactorsAssociated1990a} and refined through subsequent research, proposes that monarchs select overwintering sites based on a specific suite of microclimatic conditions: protection from freezing temperatures, access to moisture for maintaining water balance, filtered sunlight for behavioral thermoregulation, and critically, shelter from wind. These factors interact to create what researchers term the "Goldilocks zone"---conditions that are neither too hot nor too cold, too wet nor too dry, too bright nor too dark, and crucially, not too windy.

\citet{Masters1988_ACNENTPT} demonstrated the physiological basis for these requirements in Mexican overwintering sites, showing that monarchs employ complex thermoregulatory behaviors including shivering to raise thoracic temperatures for morning activity and adopting sun-minimizing postures to avoid overheating and rapid lipid depletion. In California groves, similar behavioral patterns emerge, with monarchs forming dense clusters during cool periods to reduce surface area exposure and dispersing during warm, sunny conditions to avoid overheating. However, \citet{Saniee2022_3VN7I68M} challenged the uniformity of the microclimate hypothesis, finding that microclimatic attributes in aggregation locations vary significantly with latitude and that suitable conditions exist throughout large portions of each grove, suggesting that our understanding of habitat selection remains incomplete.

Among the various abiotic factors influencing monarch overwintering habitat, wind has emerged as potentially the most critical yet remains the least understood. \citet{leongMicroenvironmentalFactorsAssociated1990a}'s seminal work first identified wind avoidance as ``the key environmental factor that affects roosting behavior,'' based on observations that monarchs consistently selected roost trees in areas with minimal wind exposure. Subsequent multivariate analyses reinforced this finding, with \citet{leongUseMultivariateAnalyses1991} demonstrating that cluster trees experienced significantly lower wind velocities than non-cluster areas within the same grove.

The wind hypothesis reached its most influential articulation in \citet{leongEvaluationManagementCalifornia2016}'s management guidelines, which explicitly stated that winds exceeding 2 m/s constitute ``disruptive winds'' that physically dislodge butterflies from clusters or trigger energetically costly escape responses. According to this framework, exposure to winds above this threshold forces monarchs to expend critical energy reserves needed for winter survival while simultaneously increasing predation risk and potentially causing permanent abandonment of roost sites. This 2 m/s threshold has become canonical in monarch conservation, directly informing habitat management decisions throughout California.

The influence of Leong's wind hypothesis on conservation practice cannot be overstated. The Xerces Society incorporated the 2 m/s threshold into their management guidelines, which serve as the primary reference for land managers throughout California. % TODO: Add specific Xerces management guide citation if available
Consulting firms like Althouse and Meade Inc. and Creekside Science now routinely conduct wind assessments as part of habitat evaluations, explicitly measuring areas that provide protection from winds exceeding 2 m/s \citep{althouse&meadeinc.EllwoodMesaSperling2023}. Grove restoration projects prioritize windbreak establishment, selective tree planting to reduce wind exposure, and careful vegetation management to maintain wind protection---all based on the assumption that winds above 2 m/s disrupt monarch aggregations.

Yet despite its widespread adoption and influence on millions of dollars in conservation investments, the disruptive wind hypothesis rests entirely on correlative evidence. Leong's original studies compared wind measurements between occupied and unoccupied trees but never directly observed butterfly responses to wind exposure. No study has documented whether monarchs actually abandon roosts when winds exceed 2 m/s, whether such abandonment represents temporary displacement or permanent desertion, or whether the presumed energetic costs actually occur. This empirical gap is particularly troubling given that wind patterns in coastal California groves are complex and highly variable, with gusts, turbulence, and directional changes that may affect monarchs differently than steady winds.

Furthermore, testing the wind hypothesis requires careful consideration of confounding factors. As ectotherms, monarchs depend entirely on environmental heat sources for activity, with solar radiation triggering departures for water-seeking, nectaring, or searching for alternative roost sites. The effectiveness of solar radiation in warming butterflies depends on ambient temperature---direct sunlight enables flight at lower intensities on warm days than cold ones. Wind itself affects this thermal balance by increasing convective heat loss, potentially lowering the temperature at which butterflies can achieve flight. Without accounting for these thermal interactions, any observed correlation between wind and monarch departures could be spurious, reflecting thermal effects rather than mechanical disruption.

This study provides the first direct empirical test of whether wind disrupts overwintering monarch butterflies through continuous observation of butterfly behavior paired with simultaneous environmental monitoring. Using high-resolution time-lapse photography to quantify monarch abundance changes within occupied roosts at Vandenberg Space Force Base, combined with on-site measurements of wind speed, temperature, humidity, and solar radiation, we can finally evaluate whether the foundational assumption underlying current management practices is valid.

We tested the disruptive wind hypothesis through a series of hierarchical analyses, each building upon previous findings. First, we hypothesized that wind acts as a disruptive force to overwintering monarch butterflies when considered alongside other environmental factors. If true, we predict that wind will emerge as a significant predictor of abundance changes in information-theoretic model selection, with monarch abundance decreasing as maximum wind speeds increase.

Second, we hypothesized that wind becomes disruptive above the specific threshold of 2 m/s proposed by \citet{leongEvaluationManagementCalifornia2016}. If this threshold represents a meaningful biological boundary, we predict that monarch abundance will show a step-change response, declining significantly at roosts experiencing winds exceeding 2 m/s compared to calmer conditions.

Third, we hypothesized that wind's disruptive effects scale with intensity beyond the threshold. If mechanical disruption increases with wind force, we predict proportionally greater decreases in monarch abundance as maximum wind speeds increase above 2 m/s, reflecting the increasing energetic costs of maintaining position in stronger winds.

By addressing these hypotheses with direct observational data while controlling for thermal conditions and other environmental variables, this study will determine whether wind truly represents a primary limiting factor for overwintering monarchs or whether current management practices rest on untested assumptions. The results have immediate implications for habitat management decisions affecting one of North America's most iconic and imperiled species.