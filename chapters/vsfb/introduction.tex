\usepackage{hyperref}
\usepackage{longtable}

\chapter{Introduction}
\label{ch:introduction}

\section{Introduction Outline}

The distribution and survival of invertebrate species are governed by a complex interplay of biotic and abiotic factors. For many insects, abiotic conditions such as temperature, humidity, and wind can be primary drivers of behavior, physiology, and habitat selection. Wind, in particular, can influence everything from dispersal and migration to foraging efficiency and predation risk.

The monarch butterfly (\textit{Danaus plexippus}) presents an excellent system for studying these dynamics. Its diverse life history, which includes distinct breeding and overwintering phases, exposes it to a unique suite of environmental challenges. As a charismatic and threatened species, understanding the factors that limit monarch populations is not only of scientific interest but also critical for effective conservation.

The overwintering stage, when monarchs form dense aggregations in coastal California and central Mexico, is a period of immense physiological stress where abiotic factors are paramount. During this time, the "microclimate hypothesis" suggests that monarchs select groves that provide a specific suite of conditions necessary for survival: protection from freezing temperatures, access to moisture, dappled sunlight for thermal regulation, and shelter from wind.

Among the abiotic factors shaping monarch overwintering habitat, wind has emerged as potentially critical yet remains poorly understood. Leong's influential work (2016) proposed that wind acts as a primary determinant of habitat suitability, specifically asserting that winds exceeding 2 m/s physically disrupt monarch clusters. According to this "disruptive wind hypothesis," such winds dislodge butterflies from their roosts or trigger escape responses, forcing monarchs to expend critical energy reserves needed for winter survival while increasing predation exposure. This 2 m/s threshold has become dogma in monarch conservation, directly informing management guidelines adopted by organizations including the Xerces Society.

Yet despite its widespread influence on conservation practice, Leong's hypothesis rests on indirect evidence. The original conclusions derived from comparing wind measurements between occupied and unoccupied roost trees. An observational approach that cannot establish causation or demonstrate actual butterfly responses to wind exposure. No study has directly observed whether monarchs actually abandon roosts when winds exceed 2 m/s, whether such abandonment represents temporary displacement or permanent desertion, or whether the presumed energy costs actually occur. This empirical gap is remarkable given that habitat management decisions affecting millions of conservation dollars assume these wind effects are real.

Testing the disruptive wind hypothesis requires isolating wind's effects from confounding environmental drivers. As ectotherms, monarchs depend on solar radiation for flight, with sunlight exposure triggering departures for foraging, water-seeking, or mate-searching. The warming effect of solar radiation itself depends on ambient temperature—direct sunlight enables flight at lower intensities on warm days than cold ones. Without accounting for this temperature-sunlight interaction, any observed correlation between wind and monarch departures could be spurious. Furthermore, wind itself is multidimensional: its potential disruption may depend not only on average speed but also on consistency (sustained winds versus calm periods) and turbulence (gustiness). A meaningful test must therefore examine how different wind characteristics influence monarch behavior while controlling for thermal conditions.

This study provides the first direct, empirical test of whether wind disrupts overwintering monarch butterflies. Using continuous time-series observations of butterfly behavior within an occupied roost, paired with simultaneous monitoring of wind conditions and thermal variables, we can finally determine whether the foundational assumption underlying current management practices is valid.

We tested whether wind disrupts overwintering monarch butterflies through a series of hierarchical hypotheses, each building upon the previous findings.

First, we hypothesized that wind acts as a disruptive force to overwintering monarch butterflies. If true, we predict that monarch abundance at roosts will decrease when exposed to disruptive winds.

Second, we hypothesized that wind becomes disruptive above a specific threshold of 2 m/s. If this threshold represents a meaningful biological boundary, we predict that monarch abundance will decline at roosts experiencing winds exceeding 2 m/s.

Third, we hypothesized that wind's disruptive effects scale with intensity. If disruption increases with wind speed, we predict proportionally greater decreases in monarch abundance as wind speeds rise above the threshold.