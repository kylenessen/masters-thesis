\usepackage{hyperref}
\usepackage{longtable}

\chapter{Introduction}
\label{ch:introduction}

The distribution and survival of invertebrate species are governed by a complex interplay of biotic and abiotic factors. Biotic interactions shape community dynamics through competition for resources, predation pressure, parasitism, and mutualistic relationships. For example, predators can exert strong indirect effects on prey populations through behavioral avoidance strategies that alter habitat use and activity patterns \citep{blois-heulinDirectIndirectEffects1990}. Parasites can manipulate host behavior to enhance their own transmission, often through neuromodulatory mechanisms that alter microhabitat selection \citep{laffertyComparingMechanismsHost2013}. Recent work using DNA metabarcoding has confirmed that predator size and species identity are key drivers structuring highly diverse terrestrial invertebrate communities \citep{miller-ter_kuilePredatorpreyInteractionsTerrestrial2022}. While these biotic interactions profoundly influence community assembly, abiotic conditions often set the fundamental limits determining where invertebrates can persist. Temperature, Humidity, Solar radiation, diurnal patterns, precipitation examples here. Among these abiotic factors, wind emerges as a particularly complex environmental force shaping invertebrate ecology.

Wind influences invertebrate behavior and ecology through diverse mechanisms that vary across species and life stages. Some species exploit wind for dispersal, such as spiders that exhibit heritable variation in ballooning motivation based on wind velocity thresholds \citep{bonteHeritabilitySpiderBallooning2007}. Other invertebrates respond to wind as an environmental stressor, with herbivorous larvae moving to leeward sides of leaves to maintain feeding efficiency when exposed to turbulent conditions \citep{leonardExposureWindAlters2016}. Cockroaches demonstrate clear behavioral thresholds, shifting from upwind orientation to crosswind escape when wind speeds exceed specific velocities \citep{bellSearchAnemotacticOrientation1979}. Beyond simple average speeds, wind's effects depend on its multidimensional characteristics including consistency, gustiness, and turbulence patterns. Understanding long-distance biological transport through air requires knowledge of turbulent deviations from mean flow, not just average conditions \citep{nathanLongdistanceBiologicalTransport2005}. Species-specific thresholds trigger distinct behavioral responses, from foraging cessation to microhabitat shifts to active dispersal. This complexity in wind-organism interactions becomes particularly evident when examining species that must navigate multiple life stages with different environmental exposures.

Many invertebrates exploit habitats where other organisms, particularly plants, modify local abiotic conditions to create more favorable microclimates. Forest canopies exemplify this phenomenon by moderating environmental extremes within their three-dimensional structure. Trees reduce wind speeds through physical baffling, with dense canopies capable of reducing wind penetration by over 90\% compared to open areas. Light exposure becomes filtered and heterogeneous, creating a mosaic of sun flecks and shade that shifts throughout the day. Temperature fluctuations are buffered, with forest interiors experiencing cooler maximum temperatures during the day and warmer minimum temperatures at night compared to adjacent open areas. Humidity remains elevated through reduced evaporation and continuous transpiration from foliage. This moderation of environmental conditions by vegetation structure creates distinct microclimates that can differ dramatically from regional conditions, enabling species to persist in areas that would otherwise exceed their physiological tolerances. The creation and exploitation of these modified environments is fundamental to understanding habitat selection patterns in many invertebrate species.

The monarch butterfly (\textit{Danaus plexippus}) presents an exceptional system for studying how abiotic factors, particularly wind, influence invertebrate ecology across complex life histories. This species undergoes one of nature's most remarkable migrations, with individuals traveling thousands of kilometers between breeding and overwintering grounds. Throughout their annual cycle, monarchs experience dramatically different environmental conditions: from temperate breeding habitats to subtropical overwintering sites, from open meadows to dense forest groves, from solitary existence to massive aggregations. Their clear behavioral responses to temperature, light, wind, and other abiotic factors have been documented across multiple life stages. This combination of complex life history, long-distance movement, and observable responses to environmental conditions makes monarchs an ideal model for examining how invertebrates navigate heterogeneous abiotic landscapes.

Individual monarch development proceeds through complete metamorphosis with each stage exhibiting specific thermal requirements and tolerances. Temperature governs development rates from egg through adult emergence, with development accelerating within physiological limits as temperatures increase. During the breeding phase, monarchs produce multiple generations per year, with three to four generations typical in northern portions of their range. Each generation exploits milkweed (\textit{Asclepias} spp.) as larvae spread across an expanding geographic range following spring warming. This obligate relationship with milkweed constrains breeding opportunities, as larvae require these plants both for nutrition and for sequestering cardiac glycosides that provide chemical defense against predators. Milkweed distribution and phenology, in turn, depend on temperature and precipitation patterns. During favorable conditions, rapid reproduction enables population growth, with individual females capable of laying hundreds of eggs. Temperature limits both larval development and adult flight activity, while precipitation affects milkweed quality and nectar availability from adult food sources. Photoperiod provides critical seasonal cues, signaling when environmental conditions will soon become unsuitable for continued breeding.

As autumn approaches, decreasing photoperiod and cooling temperatures trigger a remarkable physiological transformation in the final summer generation. These environmental cues, combined with declining milkweed quality, induce reproductive diapause through juvenile hormone suppression. This hormonal shift halts reproductive development and initiates a suite of changes that create a "super generation" adapted for long-distance migration and extended survival. Monarchs accumulate lipids up to 125\% of their lean body weight, storing energy reserves that must sustain them through migration and months of overwintering. Their lifespan extends from the typical 2-5 weeks of breeding adults to 8-9 months. Body size increases, with larger wings providing enhanced flight capacity. Perhaps most remarkably, oriented flight behavior emerges, enabling navigation across thousands of kilometers to specific overwintering locations. These coordinated changes prepare monarchs for the energetic demands of migration and the extended period of reproductive dormancy during overwintering.

Overwintering represents a critical phase in the monarch annual cycle, driven by the unsuitability of breeding habitat during winter months. Persistent freezing temperatures across northern breeding ranges would kill adults, while milkweed senesces and disappears from the landscape, making reproduction impossible even if adults could survive. Eastern monarch populations migrate to high-elevation Oyamel fir forests in central Mexico, where tens of millions of individuals concentrate in forest patches totaling less than 30 hectares. Western populations migrate to coastal California sites, primarily roosting in introduced eucalyptus groves but also utilizing native conifers and other trees. The overwintering period extends 4-5 months, during which monarchs must survive exclusively on lipid reserves accumulated during autumn migration. Survival requires avoiding multiple threats including predation, freezing, and desiccation while maintaining energy balance without significant feeding opportunities. To meet these challenges, monarchs form dense aggregations containing thousands to millions of individuals. Clustering reduces individual surface area exposure to environmental extremes, provides thermal benefits through reduced convective heat loss, and offers protection from predators through dilution effects and collective vigilance. However, storms can devastate entire aggregations, temperature fluctuations can trigger premature lipid depletion, and concentrated populations become vulnerable to disease transmission and localized habitat degradation.

Early research on monarch overwintering sites in the 1970s and 1980s focused primarily on understanding the thermal benefits of clustering behavior, particularly in Mexican overwintering sites where freezing represents a major mortality source. Researchers recognized that forest structure played a crucial role in protecting butterflies from environmental extremes. Building on these observations, Leong synthesized field observations from California overwintering sites into a formal "microclimate hypothesis" in 1990 \citep{leongMicroenvironmentalFactorsAssociated1990a}. This hypothesis proposed that monarchs select groves providing a specific envelope of suitable abiotic conditions. The original parameters included mild temperatures averaging 13°C to keep butterflies below their flight threshold and minimize metabolic expenditure, high humidity to prevent desiccation during months without drinking water, dappled sunlight allowing thermoregulation without triggering dispersal from clusters, and critically, protection from wind speeds exceeding 2 m/s to prevent physical disruption of aggregations. The underlying principle suggested that all these conditions work synergistically to minimize metabolic expenditure, thereby conserving precious lipid reserves. This conceptual framework quickly gained widespread adoption, becoming the foundation for management guidelines at overwintering sites throughout California.

Over the subsequent decades, empirical research has systematically challenged most components of the original microclimate hypothesis, revealing a more complex and scale-dependent pattern of habitat selection. The temperature component has proven particularly problematic. \citet{sanieeHierarchyScaleInfluence2022} found no significant temperature differences within groves at the scale of tens of meters, contradicting the assumption that monarchs select specific thermal microsites. Instead, temperature varies predictably with latitude across California's overwintering range, spanning an 8-18°C gradient from south to north. This suggests monarchs exhibit flexible thermal tolerance rather than requiring a fixed temperature regime. The humidity component has similarly failed empirical tests, with no meaningful variation detected at the grove scale and vapor pressure deficit showing little variation among roost sites. Perhaps most significantly, \citet{fisherClimaticNicheModel2018} demonstrated that regional macroclimate variables, particularly minimum December temperature, better predict grove occurrence than local microclimate conditions. This shift from microclimate to macroclimate causation fundamentally challenges the scale at which we understand habitat selection. Among the original parameters, only light exposure has received consistent empirical support. \citet{weissForestCanopyStructure1991} documented that permanent overwintering sites cluster around approximately 20\% canopy openness, while transient and abandoned sites show much greater variability. \citet{sanieeHierarchyScaleInfluence2022} confirmed significant light differences at actual butterfly clustering locations within groves. Yet remarkably, the wind protection component, despite being invoked as critical for management, has never received direct empirical testing in over 30 years since its proposal.

The development of the disruptive wind hypothesis followed a trajectory of increasingly strong claims based solely on correlational observations. Leong's initial 1990 work noted that occupied trees experienced lower wind speeds than unoccupied trees \citep{leongMicroenvironmentalFactorsAssociated1990a}. Subsequent multivariate analysis in 1991 reinforced this pattern, showing wind speed as a significant variable distinguishing occupied from unoccupied locations. By 1999, restoration success at Los Osos was attributed primarily to enhanced wind protection. In 2004, Leong elevated wind to a primary importance in site selection. Finally, in 2016, management guidelines codified the specific 2 m/s threshold as a critical standard \citep{leongEvaluationManagementCalifornia2016}. The hypothesis specifically asserts that winds exceeding 2 m/s physically dislodge butterflies from clusters, trigger energetically costly escape responses, and create unsuitable overwintering conditions. However, these conclusions rest entirely on correlational observations comparing occupied and unoccupied sites. This methodology cannot establish causation, as multiple environmental variables covary in nature. No study has documented whether monarchs actually abandon roosts when winds exceed the threshold, whether any abandonment is temporary or permanent, or whether presumed energetic costs actually occur. The analysis also ignored wind complexity, considering only average speeds while overlooking gustiness, turbulence, and temporal variability that may be equally or more important. Despite these limitations, the 2 m/s threshold has been adopted in federal and state management guidelines, shapes restoration designs at sites like Ellwood Mesa and Pismo Beach, and influences allocation of millions of conservation dollars for windbreak plantings.

The urgency for evidence-based monarch conservation has intensified with catastrophic population declines. Western monarch populations have declined to less than 5\% of 1980s abundance levels, with some analyses suggesting even steeper declines exceeding 95\% \citep{peltonWesternMonarchPopulation2019}. The population faces quasi-extinction risk, having crossed threshold levels below which stochastic events could eliminate remaining individuals \citep{schultzCitizenScienceData2017}. These declines prompted the U.S. Fish and Wildlife Service to propose federal listing as a threatened species with critical habitat designation in 2024 \citep{usfishEndangeredThreatenedWildlife2024}. While the total monarch breeding range spans much of North America, making comprehensive management logistically impossible, overwintering sites represent a critical bottleneck comprising less than 0.001\% of the total range. California's overwintering habitat is confined to a narrow 1.6 km coastal strip where specific climatic conditions occur. The concentration of the entire western population into approximately 400 sites for 4-5 months creates both vulnerability and opportunity. Each surviving female can lay 400+ eggs, providing exponential growth potential if overwintering survival is high. The starting population size after winter directly determines breeding season success. These sites also provide the only reliable opportunity for population monitoring through standardized counts. Over 50\% of overwintering habitat occurs on state park lands, enabling coordinated management through established partnerships. Current management strategies rely heavily on assumptions from the microclimate hypothesis, with special emphasis on achieving wind protection below the 2 m/s threshold. Major restoration investments totaling millions of dollars focus on establishing windbreaks and enhancing canopy cover based on these untested assumptions.

The critical gap in monarch conservation science is that the wind disruption hypothesis drives management decisions affecting millions of conservation dollars yet has never been empirically tested. The hypothesis has been accepted as fact for three decades without any study establishing causal relationships between wind exposure and butterfly behavior. Correlational observations comparing occupied and unoccupied sites cannot determine causation because multiple environmental factors covary in natural settings. Confounding variables must be controlled to isolate wind effects: solar radiation triggers thermoregulation independent of wind, temperature affects activity thresholds regardless of wind exposure, time of day creates predictable activity patterns, and social dynamics influence clustering behavior. Previous analyses considered only average wind speeds, ignoring the complexity of wind characteristics that may determine biological impacts. Maximum speeds may trigger escape responses even if averages remain low, consistency versus gustiness likely affects energy expenditure differently, and turbulence patterns may be more disruptive than laminar flow at the same speed. \citet{nathanLongdistanceBiologicalTransport2005} emphasized that understanding biological responses to wind requires examining deviations from mean flow, not just averages. With western monarch populations facing potential extinction, the stakes for evidence-based management have never been higher. Ineffective restoration based on untested assumptions wastes limited conservation resources and time that declining populations cannot afford.

To our knowledge, this study provides the first direct empirical test of whether wind disrupts overwintering monarch butterflies. Our primary objective was to evaluate the foundational 2 m/s wind disruption threshold that has guided three decades of conservation practice. We employed continuous monitoring at 30-minute intervals throughout the overwintering season, simultaneously measuring wind speed, temperature, and solar radiation at butterfly clustering locations. This approach enabled direct observation of butterfly responses to changing environmental conditions while controlling for confounding variables. We analyzed the data using an information-theoretic framework that compared multiple competing models to identify the strongest predictors of butterfly movement.

We tested three hierarchical hypotheses about wind effects on monarch clustering behavior. First, we hypothesized that wind, alongside other environmental factors (temperature, solar exposure, time of day), predicts changes in butterfly abundance at clusters. If true, we predict that wind will emerge as a significant predictor in model selection, with higher wind speeds associated with decreased monarch abundance. Second, we hypothesized that wind becomes specifically disruptive above the 2 m/s threshold proposed by Leong. If this threshold represents a meaningful biological boundary, we predict a discontinuous response in butterfly abundance, with stable clusters below 2 m/s and increasing departures above this speed. Third, we hypothesized that wind's disruptive effects scale with intensity above the threshold. If true, we predict a dose-response relationship where progressively higher wind speeds cause proportionally greater reductions in cluster abundance. The results of these tests will provide the evidence-based foundation necessary for effective conservation of remaining western monarch populations.