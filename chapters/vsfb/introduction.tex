\chapter{Introduction}
\label{ch:introduction}

\section{Hypotheses and Predictions}

\subsection*{Hypotheses}

\begin{enumerate}
    \item[\textbf{H1} (Loose Wind Hypothesis):] Higher wind speeds are negatively correlated with monarch butterfly abundance at overwintering sites, as increased wind causes monarchs to leave their roosting locations.
    \item[\textbf{H2} (Kingston Leong Threshold Hypothesis):] Monarch butterflies abandon overwintering sites when wind speeds exceed the critical threshold of 5~mph (2.2~m/s), with longer durations above this threshold resulting in greater site abandonment.
    \item[\textbf{H3} (Habitat Suitability Hypothesis):] Following exposure to wind speeds exceeding 5~mph while monarchs are present, butterflies will not return to previously occupied sites, indicating permanent abandonment after high-wind events.
    \item[\textbf{H4} (Sunlight-Temperature Hypothesis):] Overwintering monarch butterflies modify their clustering behavior in response to direct sunlight exposure, with this effect moderated by ambient temperature, as monarchs are sensitive to overheating risks.
\end{enumerate}

\subsection*{Predictions}

\begin{enumerate}
    \item[\textbf{P1}:] We predict a significant negative correlation between wind speed measurements and changes in monarch abundance, where higher wind speeds correspond to decreases in butterfly counts at monitoring sites.
    \item[\textbf{P2}:] We predict that the duration of time wind speeds exceed 2.2~m/s will be significantly associated with negative changes in monarch abundance, with a dose-response relationship between exposure duration and abundance decrease.
    \item[\textbf{P3}:] We predict that morning abundance counts will remain near zero at sites following days when both monarchs were present and wind speeds exceeded 5~mph, with the predictor variable ``days since last threshold exceeded'' showing no recovery in abundance for values greater than zero.
    \item[\textbf{P4}:] We predict a significant interaction between sunlight exposure proportion and ambient temperature on changes in monarch abundance, where the combination of high direct sunlight exposure and elevated temperatures will produce the greatest negative changes in butterfly counts, particularly during morning observation periods when clusters are most likely to disband.
\end{enumerate}