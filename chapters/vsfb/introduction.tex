\usepackage{hyperref}
\usepackage{longtable}

\chapter{Introduction}
\label{ch:introduction}

The distribution and survival of invertebrate species are governed by a complex interplay of biotic and abiotic factors. For many insects, abiotic conditions such as temperature, humidity, and wind can be primary drivers of behavior, physiology, and habitat selection. Wind, in particular, can influence everything from dispersal and migration to foraging efficiency and predation risk, acting as both a facilitator and a significant environmental stressor. The monarch butterfly (\textit{Danaus plexippus}) presents an exceptional system for studying these dynamics, as its complex life history exposes it to a unique suite of environmental challenges across distinct seasonal phases.

As a charismatic and threatened species, understanding the factors that limit monarch populations is critical for effective conservation. Western monarch populations have declined by more than 95\% from their historic highs, prompting their listing as ``Warranted But Precluded'' for federal endangered status \citep{EX49BGFN}. A primary threat identified in this listing is the degradation of overwintering sites in California, a critical bottleneck in the monarch life cycle where dense aggregations must survive for months on fixed energy reserves while exposed to environmental stressors.

For decades, management of these crucial sites has been guided by the ``microclimate hypothesis,'' which suggests that monarchs select groves providing a specific suite of abiotic conditions: mild temperatures, high humidity, dappled sunlight for thermoregulation, and critically, shelter from wind \citep{leongMicroenvironmentalFactorsAssociated1990a}. Among these factors, wind has been identified as a key determinant of habitat suitability. Influential work by Leong proposed the ``disruptive wind hypothesis,'' which asserts that winds exceeding a threshold of 2 m/s physically dislodge butterflies from their clusters or trigger energetically costly escape responses \citep{leongEvaluationManagementCalifornia2016}.

This 2 m/s threshold has become dogma in monarch conservation, directly informing management guidelines adopted by federal and state agencies and shaping restoration plans affecting millions of conservation dollars. Yet despite its widespread influence, the disruptive wind hypothesis is not a tested hypothesis but rather an inference resting on indirect, correlational evidence. Leong's original conclusions were derived from comparing wind measurements between occupied and unoccupied trees---an observational approach that cannot establish causation or demonstrate actual butterfly responses to wind exposure. To date, no study has empirically documented whether monarchs abandon roosts when winds exceed this threshold, if such abandonment is temporary or permanent, or if the presumed energetic costs actually occur. This analytical gap is remarkable given the hypothesis's foundational role in current conservation practice.

A direct test of this hypothesis requires isolating wind's effects from confounding environmental drivers. As ectotherms, monarchs depend on solar radiation and ambient temperature to reach the thermal threshold required for flight, and these factors can trigger departures for foraging or water-seeking independent of wind. Furthermore, wind itself is multidimensional; its potential for disruption may depend not only on average speed but also on its consistency and gustiness. A meaningful test must therefore examine how different wind characteristics influence monarch behavior while controlling for these thermal conditions.

This study provides the first direct, empirical test of whether wind disrupts overwintering monarch butterflies. Using continuous time-series observations of butterfly cluster abundance paired with simultaneous, on-site monitoring of wind, temperature, and solar radiation, we can finally evaluate whether the foundational assumption underlying current management practices is valid. We tested the disruptive wind hypothesis through a series of hierarchical analyses.

First, we hypothesized that wind acts as a disruptive force to overwintering monarch butterflies. If true, we predict that monarch abundance at roosts will decrease when exposed to disruptive winds.

Second, we hypothesized that wind becomes disruptive above a specific threshold of 2 m/s. If this threshold represents a meaningful biological boundary, we predict that monarch abundance will decline at roosts experiencing winds exceeding 2 m/s.

Third, we hypothesized that wind's disruptive effects scale with intensity. If disruption increases with wind speed, we predict proportionally greater decreases in monarch abundance as wind speeds rise above the threshold.