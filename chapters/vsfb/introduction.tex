\usepackage{hyperref}
\usepackage{longtable}

\chapter{Introduction}
\label{ch:introduction}

The distribution and survival of invertebrate species are governed by a complex interplay of biotic and abiotic factors. While biotic interactions profoundly influence community assembly in part through predation, competition, and parasitism \citep{blois-heulinDirectIndirectEffects1990,laffertyComparingMechanismsHost2013,miller-terkuilePredatorPreyInteractions2022}, abiotic conditions often set the fundamental limits determining where invertebrates can persist. Temperature constrains invertebrate physiology across all latitudes, from Antarctic midges (\textit{Belgica antarctica}) that survive freeze-thaw cycles through cryoprotective dehydration \citep{everattResponsesInvertebratesTemperature2015}, to desert land snails (\textit{Sphincterochila boissieri}) that tolerate 55°C through metabolic downregulation \citep{schweizerSnailsSunStrategies2019}. Rising temperatures mechanistically increase metabolic rates and respiratory water loss, creating compound stress that limits activity windows \citep{chownWaterLossInsects2011}. Solar radiation drives both direct physiological impacts and behavioral responses. Army ants demonstrate habitat-specific evolution of thermal maxima tied to insolation exposure \citep{baudierExtremeInsolationClimatic2018}, while terrestrial gastropods climb vertical surfaces to escape ground-level heat and evolve reflective pigmentation in exposed populations \citep{schweizerSnailsSunStrategies2019}. Among these abiotic factors, wind emerges as a particularly complex environmental force shaping invertebrate ecology through remarkably diverse mechanisms.

Wind influences invertebrate behavior and ecology through diverse mechanisms that vary across species, exhibiting remarkably specific wind detection thresholds and behavioral responses. Cockroaches perceive air currents as low as 0.015--0.03 m/s, with \textit{Periplaneta americana} demonstrating predictable orientation shifts from upwind movement at low velocities to downwind escape at higher speeds \citep{bellSearchAnemotacticOrientation1979}. Mosquito flight activity shows sensitivity to even light winds, with trap catches declining 75\% when wind speeds approach their normal flight velocity of 1.0 m/s, effectively grounding entire populations \citep{bidlingmayerEffectWindVelocity1974}. Some species exploit wind for dispersal through specific behavioral triggers, such as spiders exhibiting stereotypic `tiptoe' behavior to initiate ballooning when wind conditions permit controlled trajectories \citep{bonteHeritabilitySpiderBallooning2007}. Others respond to wind as an environmental stressor requiring active avoidance. Herbivorous larvae like \textit{Uraba lugens} treat moderate winds of 3 m/s as disturbance cues, triggering increased movement to stable microsites on branches and behavioral shifts to leeward sides and abaxial leaf surfaces for protection \citep{leonardExposureWindAlters2016}.

Beyond direct physical effects, invertebrates detect wind through indirect sensory pathways. The Namib desert beetle \textit{Lepidochora discoidalis} perceives substrate vibrations at 700--1000 Hz frequencies generated when winds exceeding 5 m/s lift sand grains, using these cues to time surface foraging when detritus becomes concentrated \citep{seelySandstormsTenebrionidBeetles1997}. The katydid \textit{Copiphora brevirostris} adjusts vibratory communication to avoid wind-induced noise, concentrating signaling between 2:00 and 5:00 AM when winds are calmest and exploiting short-term lulls during gusty periods \citep{rohmerGoneWindSignal2010}. This complexity in wind-organism interactions becomes particularly evident when examining species that must navigate multiple life stages with different environmental exposures.

The monarch butterfly (\textit{Danaus plexippus}) offers an exceptional system for studying how abiotic factors, particularly wind, shape invertebrate ecology. Their complex life history, featuring both an annual mass migration and dramatic seasonal aggregations, establishes them as an ideal model for understanding invertebrate responses to environmental constraints.

North American monarchs complete an annual cycle through a remarkable multi-generational migration that links vast northern breeding ranges with small, highly specific overwintering sites. The continent is home to two populations separated by the Rocky Mountains: a larger Eastern population and a Western population. During the spring and summer, multiple short-lived generations of butterflies (2--5 weeks) breed throughout North America, with females laying eggs exclusively on milkweed (\textit{Asclepias} spp.), the larval host plant. This breeding phase builds the population as they radiate northward, but as late summer approaches, environmental cues such as shortening day length and cooling temperatures trigger the shift to the migratory generation. These migratory adults exhibit a unique phenotype: they enter a state of reproductive diapause (suspended reproduction) and possess elongated wings suited for long-distance travel.

The fall migration begins as these long-lived adults (6--8 months) orient themselves southward, flying up to thousands of kilometers, a journey fueled by energy stored as lipid reserves accumulated from nectaring flowers along their route. Navigational feats are guided by a sophisticated system that includes a time-compensated sun compass residing in the antennae, allowing them to maintain direction throughout the day. Eastern monarchs travel to the high-altitude oyamel fir (\textit{Abies religiosa}) forests in central Mexico, while the Western population migrates to overwintering groves located along the Pacific coast of California.

 During these months, monarchs must survive exclusively on lipid reserves accumulated during autumn migration while avoiding multiple threats including predation, freezing, and desiccation. Monarchs form dense clusters on tree branches, leaves, or trunks, a behavior that reduces individual surface area exposure to environmental extremes while providing thermal benefits through reduced convective heat loss. Throughout this extended period of inactivity, abiotic factors become paramount determinants of survival, with temperature, humidity, solar exposure, and wind shaping both individual behavior and population-level outcomes.

The overwintering period, lasting roughly from mid-October through mid-March, represents an important phase of the annual migration where the entire population concentrates into remarkably small forested areas. During these months, monarchs must survive exclusively on lipid reserves accumulated during autumn migration while maintaining precise thermoregulatory balance \citep{chaplinEnergyReservesMetabolic1982,mastersThermoregulatoryBehaviorAdaptations1988}. As ectotherms, monarchs face a fundamental trade-off: they must maintain body temperatures cool enough to conserve energy through metabolic suppression, yet warm enough to permit essential movement for water acquisition and predator avoidance. The flight threshold, established at 12.7-16.0°C for Mexican populations, represents the minimum thoracic temperature required for powered flight \citep{mastersThermoregulatoryBehaviorAdaptations1988}. Below this threshold, monarchs employ behavioral thermoregulation through dorsal basking, which can elevate body temperature to flight capability within 30 seconds of sun exposure, or through energetically costly shivering that consumes energy 25 times faster than resting \citep{mastersThermoregulatoryBehaviorAdaptations1988}. Conversely, temperatures above 20°C risk terminating reproductive diapause and triggering premature remigration, while direct sun exposure can rapidly elevate thoracic temperatures to 33.6°C, forcing butterflies to adopt sun-minimizing postures or engage in heat-dissipating flights \citep{barkerEffectPhotoperiodTemperature1976,hermanHormonallyMediatedEvents1985}. This delicate energetic balance means that monarchs selecting overwintering sites must find locations that minimize both freezing risk and metabolic expenditure, as flying at 22°C expends energy 31 times faster than resting in shade at the same temperature \citep{mastersThermoregulatoryBehaviorAdaptations1988}.

Building on early observations from Mexican overwintering sites that identified forest structure as critical for mitigating environmental extremes \citep{andersonFreezeProtectionOverwintering1996,calvertEffectRainSnow1983}, Leong proposed the microclimate hypothesis for western monarchs in 1990. This hypothesis posited that monarchs select groves based on a measurable and uniform microenvironmental envelope characterized by four key parameters: wind protection below 2 m/s to prevent physical dislodgement from clusters, cool temperatures maintaining reproductive diapause while avoiding freezing mortality, dappled sunlight enabling behavioral thermoregulation without triggering dispersal, and high humidity with accessible moisture to prevent desiccation \citep{leongMicroenvironmentalFactorsAssociated1990}. Subsequent work expanded these observations, documenting that sites supporting stable winter aggregations provided shelter from prevailing north-northwest winds and storm winds from the south, with butterflies abandoning locations when winds exceeded the 2 m/s threshold \citep{leongMicroenvironmentalFactorsAssociated1991,leongAnalysisPatternDistribution2004}. The hypothesis gained widespread acceptance and directly informed federal and state management guidelines, with Leong's 2016 synthesis establishing wind protection as the single most critical environmental factor governing roosting behavior \citep{leongEvaluationManagementCalifornia2016}. This framework dictated that successful overwintering sites would consistently exhibit this specific suite of conditions, implying a predictable niche that could guide habitat restoration and protection efforts \citep{peltonStateMonarchButterfly2016}.

However, recent empirical tests have challenged the assumption of a uniform microclimate niche across the overwintering range. Testing the hypothesis that monarchs select for consistent microclimatic attributes across groves, researchers found that temperature, humidity, and light conditions varied significantly with latitude and were not uniform across overwintering locations \citep{sanieeHierarchyScaleInfluence2022}. Rather than occupying a single, universal microclimate profile, the realized niche appears geographically variable, with selection occurring hierarchically across multiple spatial scales \citep{fisherClimaticNicheModel2018}. At the grove scale, monarchs favor sites near the coast at low elevations, while within groves, they utilize interior locations that buffer environmental stochasticity \citep{fisherOverwinteringMonarchButterflies2023}. This spatial plasticity in microclimate preferences, correlated with local geography, fundamentally contradicts the original hypothesis of a rigid environmental envelope \citep{sanieeHierarchyScaleInfluence2022}. Critically, while temperature and light effects have undergone empirical testing with mixed results, the wind disruption component of the hypothesis—despite its foundational role in management practices—has never been directly tested through controlled observation of butterfly responses to measured wind exposure.

The urgency for evidence-based monarch conservation has intensified with catastrophic population declines. Western monarch populations have declined to less than 5\% of 1980s abundance levels, with some analyses suggesting even steeper declines exceeding 95\% \citep{peltonWesternMonarchPopulation2019}. The population faces quasi-extinction risk, having crossed threshold levels below which stochastic events could eliminate remaining individuals \citep{schultzCitizenScienceMonitoring2017}. These declines prompted the U.S. Fish and Wildlife Service to propose federal listing as a threatened species with critical habitat designation in 2024 \citep{u.s.fishandwildlifeserviceEndangeredThreatenedWildlife2024}. While the total monarch breeding range spans much of North America, making comprehensive management logistically impossible, overwintering sites represent a critical bottleneck comprising less than 0.001\% of the total range. California's overwintering habitat is confined to a narrow 1.6 km coastal strip where specific climatic conditions occur. The concentration of the entire western population into approximately 400 sites for 4-5 months creates both vulnerability and opportunity. Each surviving female can lay 400+ eggs, providing exponential growth potential if overwintering survival is high. The starting population size after winter directly determines breeding season success. These sites also provide the only reliable opportunity for population monitoring through standardized counts. Over 50\% of overwintering habitat occurs on state park lands, enabling coordinated management through established partnerships. Current management strategies rely heavily on assumptions from the microclimate hypothesis, with special emphasis on achieving wind protection below the 2 m/s threshold. Major restoration investments totaling millions of dollars focus on establishing windbreaks and enhancing canopy cover based on these untested assumptions.

The critical gap in monarch conservation science is that the wind disruption hypothesis drives management decisions affecting millions of conservation dollars yet has never been empirically tested. The hypothesis has been accepted as fact for three decades without any study establishing causal relationships between wind exposure and butterfly behavior. Correlational observations comparing occupied and unoccupied sites cannot determine causation because multiple environmental factors covary in natural settings. Confounding variables must be controlled to isolate wind effects: solar radiation triggers thermoregulation independent of wind, temperature affects activity thresholds regardless of wind exposure, time of day creates predictable activity patterns, and social dynamics influence clustering behavior. Accordingly, a direct test must isolate wind from temperature, direct sunlight, and diurnal timing, and examine both maximum wind speed and minutes exceeding the 2 m/s threshold. With western monarch populations facing potential extinction, the stakes for evidence-based management have never been higher. Ineffective restoration based on untested assumptions wastes limited conservation resources and time that declining populations cannot afford.

To our knowledge, this study provides the first direct empirical test of whether wind disrupts overwintering monarch butterflies. Our primary objective was to evaluate the foundational 2 m/s wind disruption threshold that has guided three decades of conservation practice. We employed continuous monitoring at 30-minute intervals throughout the overwintering season, simultaneously measuring wind speed, temperature, and solar radiation at butterfly clustering locations. This approach enabled direct observation of butterfly responses to changing environmental conditions while controlling for confounding variables. We analyzed the data using an information-theoretic framework that compared multiple competing models to identify the strongest predictors of butterfly movement.

First, we hypothesized that wind, alongside other environmental factors, predicts butterfly abundance at overwintering clusters. If true, we predict that an information-theoretic approach will identify wind as a significant predictor of abundance changes, with monarch abundance decreasing when exposed to higher wind speeds.

Second, we hypothesized that wind becomes disruptive above a specific threshold of 2 m/s. If this threshold represents a meaningful biological boundary, we predict that monarch abundance will decline at roosts experiencing winds exceeding 2 m/s.

Third, we hypothesized that wind’s disruptive effects scale with intensity. If disruption increases with wind speed, we predict proportionally greater decreases in monarch abundance as wind speeds rise above the threshold.