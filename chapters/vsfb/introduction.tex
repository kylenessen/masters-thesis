\usepackage{hyperref}
\usepackage{longtable}

\chapter{Introduction}
\label{ch:introduction}

The distribution and survival of invertebrate species are governed by a complex interplay of biotic and abiotic factors. While biotic interactions profoundly influence community assembly in part through predation, competition, and parasitism \citep{blois-heulinDirectIndirectEffects1990,laffertyComparingMechanismsHost2013,miller-terkuilePredatorPreyInteractions2022}, abiotic conditions often set the fundamental limits determining where invertebrates can persist. Temperature constrains invertebrate physiology across all latitudes, from Antarctic midges (\textit{Belgica antarctica}) that survive freeze-thaw cycles through cryoprotective dehydration \citep{everattResponsesInvertebratesTemperature2015}, to desert land snails (\textit{Sphincterochila boissieri}) that tolerate 55°C through metabolic downregulation \citep{schweizerSnailsSunStrategies2019}. Rising temperatures mechanistically increase metabolic rates and respiratory water loss, creating compound stress that limits activity windows \citep{chownWaterLossInsects2011}. Solar radiation drives both direct physiological impacts and behavioral responses. Army ants demonstrate habitat-specific evolution of thermal maxima tied to insolation exposure \citep{baudierExtremeInsolationClimatic2018}, while terrestrial gastropods climb vertical surfaces to escape ground-level heat and evolve reflective pigmentation in exposed populations \citep{schweizerSnailsSunStrategies2019}. Among these abiotic factors, wind emerges as a particularly complex environmental force shaping invertebrate ecology through remarkably diverse mechanisms.

Wind influences invertebrate behavior and ecology through diverse mechanisms that vary across species, exhibiting remarkably specific wind detection thresholds and behavioral responses. Cockroaches perceive air currents as low as 0.015--0.03 m/s, with \textit{Periplaneta americana} demonstrating predictable orientation shifts from upwind movement at low velocities to downwind escape at higher speeds \citep{bellSearchAnemotacticOrientation1979}. Mosquito flight activity shows sensitivity to even light winds, with trap catches declining 75\% when wind speeds approach their normal flight velocity of 1.0 m/s, effectively grounding entire populations \citep{bidlingmayerEffectWindVelocity1974}. Some species exploit wind for dispersal through specific behavioral triggers, such as spiders exhibiting stereotypic `tiptoe' behavior to initiate ballooning when wind conditions permit controlled trajectories \citep{bonteHeritabilitySpiderBallooning2007}. Others respond to wind as an environmental stressor requiring active avoidance. Herbivorous larvae like \textit{Uraba lugens} treat moderate winds of 3 m/s as disturbance cues, triggering increased movement to stable microsites on branches and behavioral shifts to leeward sides and abaxial leaf surfaces for protection \citep{leonardExposureWindAlters2016}.

Beyond direct physical effects, invertebrates detect wind through indirect sensory pathways. The Namib desert beetle \textit{Lepidochora discoidalis} perceives substrate vibrations at 700--1000 Hz frequencies generated when winds exceeding 5 m/s lift sand grains, using these cues to time surface foraging when detritus becomes concentrated \citep{seelySandstormsTenebrionidBeetles1997}. The katydid \textit{Copiphora brevirostris} adjusts vibratory communication to avoid wind-induced noise, concentrating signaling between 2:00 and 5:00 AM when winds are calmest and exploiting short-term lulls during gusty periods \citep{rohmerGoneWindSignal2010}. This complexity in wind-organism interactions becomes particularly evident when examining species that must navigate multiple life stages with different environmental exposures.

The monarch butterfly (\textit{Danaus plexippus}) offers an exceptional system for studying how abiotic factors, particularly wind, shape invertebrate ecology. Their complex life history, featuring both an annual mass migration and dramatic seasonal aggregations, establishes them as an ideal model for understanding invertebrate responses to environmental constraints.

North American monarchs complete an annual cycle through a remarkable multi-generational migration that links vast northern breeding ranges with small, highly specific overwintering sites. The continent is home to two populations separated by the Rocky Mountains: a larger Eastern population and a Western population. During the spring and summer, multiple short-lived generations of butterflies (2--5 weeks) breed throughout North America, with females laying eggs exclusively on milkweed (\textit{Asclepias} spp.), the larval host plant. This breeding phase builds the population as they radiate northward, but as late summer approaches, environmental cues such as shortening day length and cooling temperatures trigger the shift to the migratory generation. These migratory adults exhibit a unique phenotype: they enter a state of reproductive diapause (suspended reproduction) and possess elongated wings suited for long-distance travel.

The fall migration begins as these long-lived adults (6--8 months) orient themselves southward, flying thousands of kilometers, a journey fueled by energy stored as lipid reserves accumulated from nectaring flowers along their route. Navigational feats are guided by a sophisticated system that includes a time-compensated sun compass residing in the antennae, allowing them to maintain direction throughout the day. Eastern monarchs travel to the high-altitude oyamel fir (\textit{Abies religiosa}) forests in central Mexico, while the Western population migrates to overwintering groves located along the Pacific coast of California.

The overwintering period, lasting roughly from mid-October through mid-March, represents an important phase of the annual migration where the entire population concentrates into remarkably small forested areas. During these months, monarchs must survive exclusively on lipid reserves accumulated during autumn migration while avoiding multiple threats including predation, freezing, and desiccation. Monarchs form dense clusters on tree branches, leaves, or trunks, a behavior that reduces individual surface area exposure to environmental extremes while providing thermal benefits through reduced convective heat loss. Throughout this extended period of inactivity, abiotic factors become paramount determinants of survival, with temperature, humidity, solar exposure, and wind shaping both individual behavior and population-level outcomes.

Early research on monarch overwintering sites in the 1970s and 1980s focused primarily on understanding the thermal benefits of clustering behavior, particularly in Mexican overwintering sites where freezing represents a major mortality source. Researchers recognized that forest structure played a crucial role in protecting butterflies from environmental extremes. Building on these observations, Leong synthesized field observations from California overwintering sites into a formal "microclimate hypothesis" in 1990 \citep{leongMicroenvironmentalFactorsAssociated1990}. This hypothesis proposed that monarchs select groves providing a specific envelope of suitable abiotic conditions. The original parameters included mild temperatures averaging 13°C to keep butterflies below their flight threshold and minimize metabolic expenditure, high humidity to prevent desiccation during months without drinking water, dappled sunlight allowing thermoregulation without triggering dispersal from clusters, and critically, protection from wind speeds exceeding 2 m/s to prevent physical disruption of aggregations. The underlying principle suggested that all these conditions work synergistically to minimize metabolic expenditure, thereby conserving precious lipid reserves. This conceptual framework quickly gained widespread adoption, becoming the foundation for management guidelines at overwintering sites throughout California.

Over the subsequent decades, empirical research has systematically challenged most components of the original microclimate hypothesis, revealing a more complex and scale-dependent pattern of habitat selection. The temperature component has proven particularly problematic. \citet{sanieeHierarchyScaleInfluence2022} found no significant temperature differences within groves at the scale of tens of meters, contradicting the assumption that monarchs select specific thermal microsites. Instead, temperature varies predictably with latitude across California's overwintering range, spanning an 8-18°C gradient from south to north. This suggests monarchs exhibit flexible thermal tolerance rather than requiring a fixed temperature regime. The humidity component has similarly failed empirical tests, with no meaningful variation detected at the grove scale and vapor pressure deficit showing little variation among roost sites. Perhaps most significantly, \citet{fisherClimaticNicheModel2018} demonstrated that regional macroclimate variables, particularly minimum December temperature, better predict grove occurrence than local microclimate conditions. This shift from microclimate to macroclimate causation fundamentally challenges the scale at which we understand habitat selection. Among the original parameters, only light exposure has received consistent empirical support. \citet{weissForestCanopyStructure1991} documented that permanent overwintering sites cluster around approximately 20\% canopy openness, while transient and abandoned sites show much greater variability. \citet{sanieeHierarchyScaleInfluence2022} confirmed significant light differences at actual butterfly clustering locations within groves. Yet remarkably, the wind protection component, despite being invoked as critical for management, has never received direct empirical testing in over 30 years since its proposal.

The development of the disruptive wind hypothesis followed a trajectory of increasingly strong claims based solely on correlational observations. Leong's initial 1990 work noted that occupied trees experienced lower wind speeds than unoccupied trees \citep{leongMicroenvironmentalFactorsAssociated1990}. Subsequent multivariate analysis in 1991 reinforced this pattern, showing wind speed as a significant variable distinguishing occupied from unoccupied locations. By 1999, restoration success at Los Osos was attributed primarily to enhanced wind protection. In 2004, Leong elevated wind to a primary importance in site selection. Finally, in 2016, management guidelines codified the specific 2 m/s threshold as a critical standard \citep{leongEvaluationManagementCalifornia2016}. The hypothesis specifically asserts that winds exceeding 2 m/s physically dislodge butterflies from clusters, trigger energetically costly escape responses, and create unsuitable overwintering conditions. However, these conclusions rest entirely on correlational observations comparing occupied and unoccupied sites. This methodology cannot establish causation, as multiple environmental variables covary in nature. No study has documented whether monarchs actually abandon roosts when winds exceed the threshold, whether any abandonment is temporary or permanent, or whether presumed energetic costs actually occur. The analysis also ignored wind complexity, considering only average speeds while overlooking gustiness, turbulence, and temporal variability that may be equally or more important. Despite these limitations, the 2 m/s threshold has been adopted in federal and state management guidelines, shapes restoration designs at sites like Ellwood Mesa and Pismo Beach, and influences allocation of millions of conservation dollars for windbreak plantings.

The urgency for evidence-based monarch conservation has intensified with catastrophic population declines. Western monarch populations have declined to less than 5\% of 1980s abundance levels, with some analyses suggesting even steeper declines exceeding 95\% \citep{peltonWesternMonarchPopulation2019}. The population faces quasi-extinction risk, having crossed threshold levels below which stochastic events could eliminate remaining individuals \citep{schultzCitizenScienceMonitoring2017}. These declines prompted the U.S. Fish and Wildlife Service to propose federal listing as a threatened species with critical habitat designation in 2024 \citep{u.s.fishandwildlifeserviceEndangeredThreatenedWildlife2024}. While the total monarch breeding range spans much of North America, making comprehensive management logistically impossible, overwintering sites represent a critical bottleneck comprising less than 0.001\% of the total range. California's overwintering habitat is confined to a narrow 1.6 km coastal strip where specific climatic conditions occur. The concentration of the entire western population into approximately 400 sites for 4-5 months creates both vulnerability and opportunity. Each surviving female can lay 400+ eggs, providing exponential growth potential if overwintering survival is high. The starting population size after winter directly determines breeding season success. These sites also provide the only reliable opportunity for population monitoring through standardized counts. Over 50\% of overwintering habitat occurs on state park lands, enabling coordinated management through established partnerships. Current management strategies rely heavily on assumptions from the microclimate hypothesis, with special emphasis on achieving wind protection below the 2 m/s threshold. Major restoration investments totaling millions of dollars focus on establishing windbreaks and enhancing canopy cover based on these untested assumptions.

The critical gap in monarch conservation science is that the wind disruption hypothesis drives management decisions affecting millions of conservation dollars yet has never been empirically tested. The hypothesis has been accepted as fact for three decades without any study establishing causal relationships between wind exposure and butterfly behavior. Correlational observations comparing occupied and unoccupied sites cannot determine causation because multiple environmental factors covary in natural settings. Confounding variables must be controlled to isolate wind effects: solar radiation triggers thermoregulation independent of wind, temperature affects activity thresholds regardless of wind exposure, time of day creates predictable activity patterns, and social dynamics influence clustering behavior. Accordingly, a direct test must isolate wind from temperature, direct sunlight, and diurnal timing, and examine both maximum wind speed and minutes exceeding the 2 m/s threshold. With western monarch populations facing potential extinction, the stakes for evidence-based management have never been higher. Ineffective restoration based on untested assumptions wastes limited conservation resources and time that declining populations cannot afford.

To our knowledge, this study provides the first direct empirical test of whether wind disrupts overwintering monarch butterflies. Our primary objective was to evaluate the foundational 2 m/s wind disruption threshold that has guided three decades of conservation practice. We employed continuous monitoring at 30-minute intervals throughout the overwintering season, simultaneously measuring wind speed, temperature, and solar radiation at butterfly clustering locations. This approach enabled direct observation of butterfly responses to changing environmental conditions while controlling for confounding variables. We analyzed the data using an information-theoretic framework that compared multiple competing models to identify the strongest predictors of butterfly movement.

First, we hypothesized that wind, alongside other environmental factors, predicts butterfly abundance at overwintering clusters. If true, we predict that an information-theoretic approach will identify wind as a significant predictor of abundance changes, with monarch abundance decreasing when exposed to higher wind speeds.

Second, we hypothesized that wind becomes disruptive above a specific threshold of 2 m/s. If this threshold represents a meaningful biological boundary, we predict that monarch abundance will decline at roosts experiencing winds exceeding 2 m/s.

Third, we hypothesized that wind’s disruptive effects scale with intensity. If disruption increases with wind speed, we predict proportionally greater decreases in monarch abundance as wind speeds rise above the threshold.