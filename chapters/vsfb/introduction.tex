\usepackage{hyperref}
\usepackage{longtable}

\chapter{Introduction}
\label{ch:introduction}

\section{Introduction Outline}

\textit{Preamble: Below is roughly the narrative direction I'm thinking for the introduction. Of course, a lot of details need to be filled in, but I try to outline the flow as best I can.}

The distribution and survival of invertebrate species are governed by a complex interplay of biotic and abiotic factors. For many insects, abiotic conditions such as temperature, humidity, and wind can be primary drivers of behavior, physiology, and habitat selection. Wind, in particular, can influence everything from dispersal and migration to foraging efficiency and predation risk.

The monarch butterfly (\textit{Danaus plexippus}) presents an excellent system for studying these dynamics. Its diverse life history, which includes distinct breeding and overwintering phases, exposes it to a unique suite of environmental challenges. As a charismatic and threatened species, understanding the factors that limit monarch populations is not only of scientific interest but also critical for effective conservation.

The overwintering stage, when monarchs form dense aggregations in coastal California and central Mexico, is a period of immense physiological stress where abiotic factors are paramount. During this time, the "microclimate hypothesis" suggests that monarchs select groves that provide a specific suite of conditions necessary for survival: protection from freezing temperatures, access to moisture, dappled sunlight for thermal regulation, and shelter from wind.

Foundational work by Kingston Leong has been central to our understanding of monarch overwintering needs, specifically identifying wind as a key determinant of habitat suitability. Leong (2016) posits that winds exceeding 2 m/s are "disruptive," causing butterflies to be either physically dislodged from their roosts or induced to flee. Such disruption is considered detrimental as it can force the butterflies to expend critical energy reserves needed to survive the winter and may increase their exposure to predation. This assertion directly informs our choice of response variable: the change in monarch abundance at a roost.

Leong's understanding of wind's impact has been highly influential, forming the basis for management guidelines by organizations like the Xerces Society. However, his conclusions were largely inferred from comparing wind measurements at occupied roost trees versus unoccupied trees, rather than from direct observation of butterfly behavioral responses to wind. In contrast to Leong's methods, this study uses direct, continuous time-series observation of behavioral responses within a single, occupied roost. To our knowledge, this relationship has never been directly and empirically tested.

Furthermore, to isolate the effect of wind, we must also account for other environmental drivers. Monarchs are ectotherms that rely on solar radiation to warm their bodies for flight. Exposure to sunlight is therefore a primary driver of activity and can trigger dispersal as individuals leave the roost to forage for nectar, seek water, or find mates. Therefore, to accurately test the disruptive wind hypothesis, we must statistically control for the confounding effects of sunlight exposure and ambient temperature. We include these as an interaction term because the effect of solar radiation on a butterfly's body temperature is not independent of the ambient temperature. Direct sunlight provides the primary thermal energy for flight, but the amount of warming required to reach a flight threshold is lower on a warm day than on a cold one, making the effects of these two variables inherently interconnected.

While this 2 m/s threshold provides a critical starting point, the nature of wind is complex. Its disruptive force may be a function of not just a single speed but also its consistency (sustained wind) and turbulence (gustiness). Therefore, a comprehensive test of the disruptive wind hypothesis requires examining these different wind characteristics.

While Leong's foundational work provides compelling evidence for wind as a determinant of overwintering habitat quality, his conclusions rest on indirect inference rather than direct observation. This study addresses this gap by directly observing monarch behavioral responses to wind within occupied roosts. We test whether wind disrupts overwintering monarchs and, if so, how different wind characteristics influence this disruption.

We tested whether wind disrupts overwintering monarch butterflies through a series of hierarchical hypotheses, each building upon the previous findings.

First, we hypothesized that wind acts as a disruptive force to overwintering monarch butterflies. If true, we predict that monarch abundance at roosts will decrease when exposed to disruptive winds.

Second, we hypothesized that wind becomes disruptive above a specific threshold of 2 m/s. If this threshold represents a meaningful biological boundary, we predict that monarch abundance will decline at roosts experiencing winds exceeding 2 m/s.

Third, we hypothesized that wind's disruptive effects scale with intensity. If disruption increases with wind speed, we predict proportionally greater decreases in monarch abundance as wind speeds rise above the threshold.

Fourth, we hypothesized that wind magnitude influences the probability of roost abandonment. If higher winds increase abandonment risk, we predict that the likelihood of monarchs completely deserting a roost will rise with increasing wind speeds.

Finally, we hypothesized that experiencing disruptive winds affects monarch site fidelity. If wind disruption influences future roost selection, we predict decreased site fidelity manifested as sustained abundance reductions at wind-disturbed roosts compared to pre-disturbance levels. 