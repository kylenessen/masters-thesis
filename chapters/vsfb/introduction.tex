\chapter{Introduction}
\label{ch:introduction}

\section{Hypotheses and Predictions}

\subsection{Wind Dispersal Hypothesis (H1)}

\textbf{Hypothesis:} Higher wind speeds are negatively correlated with monarch butterfly abundance at overwintering sites, as increased wind causes monarchs to leave their roosting locations.

\textbf{Prediction:} We predict a significant negative correlation between wind speed measurements and changes in monarch abundance, where higher wind speeds correspond to decreases in butterfly counts at monitoring sites.

\textbf{Proposed Analysis:}
We will test this hypothesis using a mixed-effects linear model with temporal autocorrelation to account for the hierarchical structure of our data and time-series dependencies. The response variable will be the change in monarch abundance calculated between consecutive observation periods, allowing for both positive (recruitment/clustering) and negative (departure/dispersal) values. Primary fixed effects will include mean wind speed (averaged across the observation period), maximum wind speed, 95th percentile wind speed, and wind speed variance, which together capture the overall wind exposure, peak wind events, extreme conditions, and gustiness experienced during each observation interval.

Beyond testing linear relationships, we recognize that H1 encompasses a broad investigation of wind effects on monarch behavior. If initial analyses suggest that wind is indeed a significant factor influencing overwintering abundance, we propose additional approaches to characterize the nature of these relationships. Specifically, we could implement segmented regression to identify potential breakpoints in wind-abundance relationships, allowing the data to reveal critical wind speeds where monarch behavior changes. Alternatively, smoothing splines or piecewise regression could capture non-linear responses to wind exposure. These threshold discovery methods would help determine whether wind effects on monarch abundance are gradual and continuous or exhibit sharp transitions at specific wind speeds.

Should H2 fail to support the Kingston Leong threshold, these exploratory threshold detection approaches might warrant development into a separate hypothesis focused specifically on identifying data-driven wind thresholds rather than testing literature-derived values.

To account for non-independence in our data structure, we will include random intercepts for camera view (to control for site-specific variation) and labeler identity (to control for observer effects in abundance estimation). Given the temporal nature of abundance measurements, we will incorporate an AR(1) autocorrelation structure to model the expected correlation between consecutive time points, which is critical for obtaining unbiased parameter estimates in time-series ecological data.

Model diagnostics will include examination of residual plots, normality assessment, and validation of the autocorrelation structure. We anticipate reporting standardized effect sizes for wind variables, 95\% confidence intervals for all parameters, and visualization of the predicted relationship between wind speed and abundance change. If threshold effects are detected, we will report the estimated breakpoint(s) with confidence intervals. A significant negative coefficient for wind speed variables would support our hypothesis that increased wind conditions drive monarch dispersal from overwintering sites.

The statistical model will be implemented in R using the \texttt{nlme} package as follows:

\begin{verbatim}
# Linear model
model_h1_linear <- lme(abundance_change ~ mean_wind_speed + max_wind_speed + 
                       wind_speed_95th + wind_speed_variance,
                       random = ~ 1 | view/labeler,
                       correlation = corAR1(form = ~ time_index | view),
                       data = monarch_wind_data,
                       method = "REML")

# Threshold discovery using segmented regression
library(segmented)
model_h1_segmented <- segmented(model_h1_linear, 
                               seg.Z = ~ mean_wind_speed,
                               npsi = 1)  # test for 1 breakpoint
\end{verbatim}

\subsection{Critical Wind Threshold Hypothesis (H2)}

\textbf{Hypothesis:} Monarch butterflies abandon overwintering sites when wind speeds exceed Kingston Leong's critical threshold of ~5 mph (2.2 m/s).

\textbf{Prediction:} We predict that exposure to wind speeds exceeding 2.2 m/s will be significantly associated with negative changes in monarch abundance.

\textbf{Proposed Analysis:}
We will test the Kingston Leong threshold hypothesis using a simple mixed-effects model that directly examines the relationship between threshold exceedance and abundance change. The response variable will be change in monarch abundance between consecutive observation periods. The primary fixed effect will be \texttt{minutes\_above\_2.2ms}, representing the total duration (in minutes) that wind speeds exceeded the Kingston threshold during each observation interval.

This focused approach tests the specific claim that 2.2~m/s represents a critical threshold for monarch site abandonment, independent of other wind characteristics. We will include the same random effects structure as H1 (random intercepts for camera view and labeler identity) and AR(1) temporal autocorrelation to account for data dependencies.

A significant negative coefficient for \texttt{minutes\_above\_2.2ms} would support the Kingston Leong threshold hypothesis, while a non-significant result would suggest that this specific threshold may not be biologically meaningful for monarch site abandonment behavior.

The statistical model will be implemented in R using the \texttt{nlme} package as follows:

\begin{verbatim}
model_h2 <- lme(abundance_change ~ minutes_above_2.2ms,
                random = ~ 1 | view/labeler,
                correlation = corAR1(form = ~ time_index | view),
                data = monarch_wind_data,
                method = "REML")
\end{verbatim}

\subsection{Site Fidelity Loss Hypothesis (H3)}

\textbf{Hypothesis:} Following exposure to wind speeds exceeding 5~mph while monarchs are present, butterflies will not return to previously occupied sites, indicating permanent abandonment after high-wind events.

\textbf{Prediction:} We predict that morning abundance counts will remain near zero at sites following days when both monarchs were present and wind speeds exceeded 5~mph, with the predictor variable ``days since last threshold exceeded'' showing no recovery in abundance for values greater than zero.

\textbf{Proposed Analysis:}
We will test this hypothesis using a logistic mixed-effects model that examines site occupancy patterns following wind threshold events. This approach directly tests the conservation claim that monarchs permanently abandon sites after experiencing unsuitable wind conditions.

We define a wind threshold event as any day when wind speeds exceed 2.2~m/s for ≥30 consecutive minutes while monarchs are present (abundance > 0). This operational definition ensures we capture sustained wind exposure during active site occupation, rather than brief gusts or events occurring at unoccupied sites. The 30-minute minimum duration aligns with our photographic sampling interval.

The response variable will be binary site occupancy status (occupied = abundance > 0, unoccupied = abundance = 0) derived from morning abundance counts to avoid confounding with same-day wind effects. We focus exclusively on the 14-day period following each threshold event, which captures both immediate and sustained abandonment patterns predicted by conventional wisdom.

The primary fixed effect will be \texttt{days\_since\_wind\_event}, representing the number of days elapsed since the threshold event occurred. According to the site abandonment hypothesis, we expect a strong negative relationship where the probability of site occupation remains near zero for all post-event time points, with no recovery pattern over the 14-day analysis window.

This conservative analytical approach uses complete abandonment (zero abundance) as the response threshold, giving the conventional wisdom hypothesis the strongest possible test. If monarchs truly abandon sites after wind exposure, this effect should be easily detectable even with limited sample sizes. We will include random intercepts for camera view and labeler identity to control for site-specific and observer effects, following the same random effects structure used in H1 and H2.

Model diagnostics will focus on examining residual patterns, assessing model convergence, and validating the binary response assumption. We will report odds ratios for the temporal predictor, 95\% confidence intervals, and visualization of predicted occupancy probability across the post-event time series. A significant negative coefficient for \texttt{days\_since\_wind\_event} would support the site abandonment hypothesis, while a non-significant result would challenge current conservation guidance regarding wind exposure and habitat suitability.

The statistical model will be implemented in R using the \texttt{glmer} function from the \texttt{lme4} package as follows:

\begin{verbatim}
library(lme4)
model_h3 <- glmer(site_occupied ~ days_since_wind_event + (1|view) + (1|labeler),
                  family = binomial,
                  data = post_event_data)
\end{verbatim}

\subsection{Thermal Regulation Hypothesis (H4)}

\textbf{Hypothesis:} Overwintering monarch butterflies modify their clustering behavior in response to direct sunlight exposure, with this effect moderated by ambient temperature, as monarchs are sensitive to overheating risks.

\textbf{Prediction:} We predict a significant interaction between sunlight exposure proportion and ambient temperature on changes in monarch abundance, where the combination of high direct sunlight exposure and elevated temperatures will produce the greatest negative changes in butterfly counts, particularly during morning observation periods when clusters are most likely to disband.

\textbf{Proposed Analysis:}
We will test this hypothesis using a mixed-effects linear model with an interaction term to examine the combined effects of sunlight exposure and ambient temperature on monarch clustering behavior. This approach directly tests the thermal regulation hypothesis that monarchs modify their behavior to avoid overheating when exposed to direct sunlight under elevated temperature conditions.

The response variable will be change in monarch abundance calculated between consecutive observation periods, allowing for both positive (clustering) and negative (dispersal) values. Our primary fixed effects will include: (1) sunlight exposure proportion, calculated as the proportion of butterflies experiencing direct sunlight relative to total butterflies present during each observation period; (2) ambient temperature, derived from temperature readings extracted via OCR from deployment photographs; and (3) their interaction term (sunlight exposure $\times$ ambient temperature), which tests our core prediction that thermal effects are most pronounced when both factors are elevated.

This sunlight exposure metric addresses the key challenge of scaling exposure measurements to population size. By calculating the proportion of butterflies in direct sunlight relative to the total butterfly population present, we obtain a continuous variable between 0 and 1. Periods with zero sunlight exposure represent times when no butterflies were positioned in sunny locations, providing natural contrast conditions for our analysis.

To control for baseline differences between morning and afternoon observation periods, we will include time period (morning vs. afternoon) as a fixed effect in our model. This approach accounts for natural temporal variation in monarch behavior while allowing us to test our core thermal regulation hypothesis. Since direct sunlight exposure occurs predominantly during morning hours in our dataset, this design controls for time-of-day effects while focusing analytical power on detecting thermal responses to sunlight exposure.

We will include the same random effects structure as H1-H3 (random intercepts for camera view and labeler identity) to control for site-specific variation and observer effects. Given the temporal nature of our data, we will incorporate AR(1) autocorrelation to model expected correlation between consecutive time points, ensuring unbiased parameter estimates in our time-series analysis.

Model diagnostics will focus on examining residual patterns, validating the autocorrelation structure, and assessing the distribution of sunlight exposure values. We will report interaction effect coefficients with 95\% confidence intervals, standardized effect sizes for all predictors, and visualization of the predicted relationship between sunlight-temperature combinations and abundance change. Results will include the estimated effect of time period to characterize baseline differences between morning and afternoon observations.

A significant negative interaction coefficient would support our thermal regulation hypothesis, indicating that monarch abundance decreases most markedly when both sunlight exposure and ambient temperature are elevated. Conversely, non-significant interaction effects would suggest that thermal regulation responses are not dependent on the combined influence of these factors, challenging current understanding of monarch thermoregulatory behavior at overwintering sites.

The statistical model will be implemented in R using the \texttt{nlme} package as follows:

\begin{verbatim}
# Thermal regulation model with period control
model_h4 <- lme(abundance_change ~ sunlight_exposure_prop * ambient_temp + period,
                random = ~ 1 | view/labeler,
                correlation = corAR1(form = ~ time_index | view),
                data = monarch_thermal_data,
                method = "REML")
\end{verbatim}