\usepackage{hyperref}
\usepackage{longtable}

\chapter{Introduction}
\label{ch:introduction}

The distribution and survival of invertebrate species are governed by a complex interplay of biotic and abiotic factors. While biotic interactions influence community assembly in part through predation, competition, and parasitism \citep{blois-heulinDirectIndirectEffects1990,laffertyComparingMechanismsHost2013,miller-terkuilePredatorPreyInteractions2022}, abiotic conditions often set the fundamental limits determining where invertebrates can persist. Temperature constrains invertebrate physiology across all latitudes, from Antarctic midges (\textit{Belgica antarctica}) that survive freeze-thaw cycles through cryoprotective dehydration \citep{everattResponsesInvertebratesTemperature2015}, to desert land snails (\textit{Sphincterochila boissieri}) that tolerate 55°C through metabolic downregulation \citep{schweizerSnailsSunStrategies2019}. Rising temperatures directly increase metabolic rates and respiratory water loss, creating compound stress that limits activity windows \citep{chownWaterLossInsects2011}. Solar radiation drives both direct physiological impacts and behavioral responses. Army ants demonstrate habitat-specific evolution of thermal maxima tied to insolation exposure \citep{baudierExtremeInsolationClimatic2018}, while terrestrial gastropods climb vertical surfaces to escape ground-level heat and evolve reflective pigmentation in exposed populations \citep{schweizerSnailsSunStrategies2019}. Among these abiotic factors, wind emerges as a particularly complex environmental force shaping invertebrate ecology through remarkably diverse mechanisms.

Wind influences invertebrate behavior and ecology through diverse mechanisms that vary across species, exhibiting remarkably specific wind detection thresholds and behavioral responses. Cockroaches perceive air currents as low as 0.015--0.03 m/s, with \textit{Periplaneta americana} demonstrating predictable orientation shifts from upwind movement at low velocities to downwind escape at higher speeds \citep{bellSearchAnemotacticOrientation1979}. Mosquito flight activity shows sensitivity to even light winds, with trap catches declining 75\% when wind speeds approach their normal flight velocity of 1.0 m/s, effectively grounding entire populations \citep{bidlingmayerEffectWindVelocity1995}. Some species exploit wind for dispersal through specific behavioral triggers, such as spiders exhibiting stereotypic `tiptoe' behavior to initiate ballooning when wind conditions permit controlled trajectories \citep{bonteHeritabilitySpiderBallooning2007}. Others respond to wind as an environmental stressor requiring active avoidance. Herbivorous larvae like \textit{Uraba lugens} treat moderate winds of 3 m/s as disturbance cues, triggering increased movement to stable microsites on branches and behavioral shifts to leeward sides and abaxial leaf surfaces for protection \citep{leonardExposureWindAlters2016}.

Beyond direct physical effects, invertebrates detect wind through indirect sensory pathways. The Namib desert beetle \textit{Lepidochora discoidalis} perceives substrate vibrations when winds exceeding 5 m/s lift sand grains, using these cues to time surface foraging when detritus becomes concentrated \citep{hanrahanEffectWindForaging1997}. The katydid \textit{Copiphora brevirostris} adjusts vibratory communication to avoid wind-induced noise, concentrating signaling between 2:00 and 5:00 AM when winds are calmest and exploiting short-term lulls during gusty periods \citep{velillaGoneWindSignal2020}. This complexity in wind-organism interactions becomes particularly evident when examining species that must navigate multiple life stages with different environmental exposures.

The monarch butterfly (\textit{Danaus plexippus}) offers an exceptional system for studying how abiotic factors, particularly wind, shape invertebrate ecology. Their complex life history, featuring both an annual mass migration and dramatic seasonal aggregations, establishes them as an ideal model for understanding invertebrate responses to environmental constraints.

North American monarchs complete an annual cycle through a remarkable multi-generational migration that links vast northern breeding ranges with small, highly specific overwintering sites \citep{browerUnderstandingMisunderstandingMigration1995}. The continent is home to two populations separated by the Rocky Mountains: a larger Eastern population and a Western population \citep{browerUnderstandingMisunderstandingMigration1995}. During the spring and summer, multiple short-lived generations of butterflies (2--5 weeks) breed throughout North America, with females laying eggs exclusively on milkweed (\textit{Asclepias} spp.), the larval host plant \citep{browerUnderstandingMisunderstandingMigration1995}. This breeding phase builds the population as they radiate northward, but as late summer approaches, environmental cues such as shortening day length and cooling temperatures trigger the shift to the migratory generation \citep{reppertDemystifyingMonarchButterfly2018}. These migratory adults exhibit a unique phenotype: they enter a state of reproductive diapause (suspended reproduction) and possess elongated wings suited for long-distance travel \citep{barkerEffectPhotoperiodTemperature1976}.

The fall migration begins as these long-lived adults (6--8 months) orient themselves southward, flying up to thousands of kilometers, a journey fueled by energy stored as lipid reserves accumulated from nectaring flowers along their route \citep{chaplinEnergyReservesMetabolic1982}. Navigational feats are guided by a sophisticated system that includes a time-compensated sun compass residing in the antennae, allowing them to maintain direction throughout the day \citep{mouritsenVirtualMigrationTethered2002,reppertDemystifyingMonarchButterfly2018}. Eastern monarchs travel to the high-altitude oyamel fir (\textit{Abies religiosa}) forests in central Mexico, while the Western population migrates to overwintering groves located along the Pacific coast of California \citep{browerUnderstandingMisunderstandingMigration1995,Urquhart1978Autumnal}.

The overwintering period, lasting roughly from mid-October through mid-March, represents an important phase of the annual migration where the entire North American population concentrates into remarkably small forested areas. During these months, monarchs must survive exclusively on lipid reserves accumulated during autumn migration while maintaining precise thermoregulatory balance \citep{chaplinEnergyReservesMetabolic1982,mastersMonarchButterflyDanaus1988}. As ectotherms, monarchs face a fundamental trade-off: they must maintain body temperatures cool enough to conserve energy through metabolic suppression, yet warm enough to permit essential daily movement \citep{mastersMonarchButterflyDanaus1988}. The flight threshold, established at 12.7-16.0°C for Mexican populations, represents the minimum thoracic temperature required for powered flight \citep{mastersMonarchButterflyDanaus1988}. Below this threshold, monarchs employ behavioral thermoregulation through dorsal basking, which can elevate body temperature to flight capability within 30 seconds of sun exposure, or through energetically costly shivering that consumes energy 25 times faster than resting \citep{mastersMonarchButterflyDanaus1988}. Conversely, temperatures above 20°C risk terminating reproductive diapause and triggering premature remigration, while direct sun exposure can rapidly elevate thoracic temperatures to 33.6°C, forcing butterflies to adopt sun-minimizing postures or engage in heat-dissipating flights \citep{barkerEffectPhotoperiodTemperature1976,barkerEffectPhotoperiodTemperature1976}. This delicate energetic balance means that monarchs selecting overwintering sites must find locations that minimize both freezing risk and elevated metabolic expenditure \citep{mastersMonarchButterflyDanaus1988}.

Building on early studies from Mexican overwintering sites that identified abiotic factors limiting overwintering butterfly success \citep{andersonFreezeprotectionOverwinteringMonarch1996,calvertEffectRainSnow1983}, Leong proposed the microclimate hypothesis for western monarchs \citep{leongMicroenvironmentalFactorsAssociated1990}. This hypothesis posited that monarchs select groves based on a measurable and uniform microenvironmental envelope characterized by four key parameters: wind protection below 2 m/s to prevent disruption to clusters, cool temperatures maintaining reproductive diapause while avoiding freezing mortality, dappled sunlight enabling behavioral thermoregulation, and high humidity with accessible moisture to prevent desiccation \citep{leongMicroenvironmentalFactorsAssociated1990}. Subsequent work expanded these observations, documenting that sites supporting stable winter aggregations provided shelter from prevailing north-northwest winds and storm winds from the south, with butterflies abandoning locations when winds exceeded the 2 m/s threshold \citep{leongUseMultivariateAnalyses1991,leong2004analysis}. The hypothesis gained widespread acceptance and directly informed federal, state, and private management guidelines, with Leong's 2016 synthesis establishing wind protection as a critical environmental factor governing roosting behavior \citep{leongEvaluationManagementCalifornia2016}. This framework dictated that successful overwintering sites would consistently exhibit this specific suite of conditions, implying a predictable niche that could guide habitat restoration and protection efforts \citep{xercessocietyStateMonarchOverwintering2016}.

However, empirical testing has gradually dismantled the microclimate hypothesis's core assumption of a uniform environmental envelope. When researchers tested whether monarchs select for consistent microclimatic attributes across groves, they found that the realized microclimate varied significantly with latitude and local geography \citep{sanieeHierarchyScaleInfluence2022}. Temperature responses proved more complex than originally proposed, with monarchs avoiding freezing temperatures but selecting the warmest of cold temperatures available at their latitude \citep{fisherClimaticNicheModel2018}. Humidity requirements showed geographic variability rather than consistency. Of the four original parameters, only light exposure showed potential support as a unifying factor, with successful overwintering sites exhibiting consistent patterns of canopy openness \citep{sanieeHierarchyScaleInfluence2022}. This geographic variability in microclimate preferences, where selection occurs hierarchically across multiple spatial scales rather than for a single environmental envelope, fundamentally contradicts Leong's original hypothesis \citep{fisherClimaticNicheModel2018,sanieeHierarchyScaleInfluence2022}. The gradual rejection of the uniform microclimate hypothesis through empirical testing suggests that monarch habitat selection may be driven by physiological and energetic constraints that vary with local environmental conditions rather than by a rigid set of universal requirements.

The wind disruption component of the hypothesis stands apart as the only major parameter that has never undergone direct empirical testing. Despite wind protection below 2 m/s being identified as a critical factor in Leong's framework \citep{leongEvaluationManagementCalifornia2016}, widely adopted in restoration efforts \citep{xercessocietyManagingMonarchsWest2018}, no study has established causal relationships between wind exposure and butterfly behavior. The original observations comparing wind measurements between occupied and unoccupied trees represent correlational evidence that cannot distinguish whether butterflies actively avoid wind or whether wind correlates with other unmeasured variables.

The absence of empirical testing for the wind disruption hypothesis becomes critically important given the catastrophic decline of western monarch populations. Between the 1980s and mid-2010s, populations collapsed by approximately 97\% \citep{schultzCitizenScienceMonitoring2017}, with population viability analyses revealing a 72\% quasi-extinction risk within 20 years \citep{schultzCitizenScienceMonitoring2017}. The winter 2018 to 2019 count recorded fewer than 30,000 monarchs, a single-year drop of 86\%, representing over 99\% decline from 1980s abundance \citep{peltonWesternMonarchPopulation2019}. The situation reached its nadir in 2020 when the Western Monarch Thanksgiving Count documented only 1,901 butterflies, the lowest number ever recorded \citep{xercessocietyWesternMonarchThanksgiving2025}. While populations showed some recovery in subsequent years, the 2024 to 2025 count of just 9,119 butterflies represents the second-lowest count on record, demonstrating the population's continued precarity \citep{xercessocietyWesternMonarchThanksgiving2025}. These dramatic declines result from multiple interacting stressors including breeding habitat loss, pesticide exposure, climate change, and critically, the loss and degradation of overwintering habitat \citep{croneWhyAreMonarch2019,peltonWesternMonarchPopulation2019}. Evidence increasingly suggests that the overwintering stage represents the most limiting phase of the monarch's annual cycle, with population declines concentrated during winter and early spring periods \citep{peltonWesternMonarchPopulation2019}. California's coastal overwintering groves, where the entire western population concentrates into remarkably small forested areas from October through March, have thus emerged as critical conservation priorities. Yet management of these critical sites continues to rely on the untested wind disruption hypothesis, with restoration efforts investing millions of conservation dollars based on the presumed necessity of maintaining wind speeds below 2 m/s \citep{althouse&meadeinc.EllwoodMesaSperling2023,xercessocietyManagingMonarchsWest2018,peltonMonarchButterflyOverwintering2020,jepsenProtectingCaliforniasButterfly2017,weissAlbanyHillMonarch2018}.

Testing the wind disruption hypothesis requires addressing a fundamental methodological challenge: isolating wind effects from confounding environmental variables that naturally covary in field settings. Solar radiation can trigger butterfly departures through direct heating, enabling flight when ambient temperatures remain below thermal thresholds \citep{mastersMonarchButterflyDanaus1988}. Temperature independently influences activity patterns, with monarchs exhibiting predictable responses as temperatures approach and exceed flight thresholds \citep{barkerEffectPhotoperiodTemperature1976}. Time of day creates inherent activity rhythms related to sun angle and thermal conditions. Furthermore, wind itself is multidimensional, characterized not only by average speed but also by gustiness and variability \citep{nathanLongdistanceBiologicalTransport2005} that could differentially affect butterfly behavior. A rigorous test must therefore control for these confounding factors while examining multiple aspects of wind exposure, including both maximum wind speeds and duration above the proposed 2 m/s threshold. With western monarch populations facing potential extinction and limited conservation resources available, the need for empirically validated management strategies has never been more urgent.

This study provides the first direct empirical test of whether wind disrupts overwintering monarch butterflies. Our primary objective was to evaluate the foundational 2 m/s wind disruption threshold that has guided over three decades of conservation practice. We employed continuous monitoring at 30-minute intervals throughout the overwintering season, simultaneously measuring wind speed, temperature, and solar radiation at butterfly clustering locations. This approach enabled direct observation of butterfly responses to changing environmental conditions while controlling for confounding variables. We analyzed the data using an information-theoretic framework that compared multiple competing models to identify the strongest predictors of butterfly movement.

First, we hypothesized that wind, alongside other environmental factors, predicts butterfly abundance at overwintering clusters. If true, we predict that an information-theoretic approach will identify wind as a significant predictor of abundance changes, with monarch abundance decreasing when exposed to higher wind speeds.

Second, we hypothesized that wind becomes disruptive above a specific threshold of 2 m/s. If this threshold represents a meaningful biological boundary, we predict that monarch abundance will decline at roosts experiencing winds exceeding 2 m/s.

Third, we hypothesized that wind’s disruptive effects scale with intensity. If disruption increases with wind speed, we predict proportionally greater decreases in monarch abundance as wind speeds rise above the threshold.