\section{Materials and Methods}

\subsection{Study Site}

Site selection followed a systematic filtering process driven by project requirements and practical constraints. The study was supported by a federal grant that mandated research be conducted on federal lands. We selected Vandenberg Space Force Base (VSFB, 34.7398°N, 120.5725°W) in Santa Barbara County, California, based on several key advantages. The base experiences mild winters with infrequent frost events, and extensive historical plantings of blue gum eucalyptus (\textit{Eucalyptus globulus}) have created suitable overwintering habitat throughout the installation. Consequently, the base contains thirty documented monarch overwintering groves, with several sites consistently ranking within the top 10\% of population counts statewide over the past decade \autocite{xercesGuideWesternMonarch2025}.  Additionally, the restricted access of this military installation provided security for long-term equipment deployment by minimizing disturbance from unauthorized individuals. 

Working in collaboration with the base's monarch conservation coordinator, we initially screened twelve locations from the thirty sites based on their documented capacity to support monarch aggregations and provision of year-round research access. This collaboration leveraged local expertise derived from managing Western Monarch Thanksgiving Count activities for multiple years \autocite{xercesGuideWesternMonarch2025}. During the study period, ten of these sites were actively monitored. However, due to low monarch populations during the 2023-2024 season, and no observed overwintering behavior in the 2024-2025 season, only two sites, Spring Canyon and UDMH, produced measurable butterfly clusters suitable for behavioral analysis.

Spring Canyon (34.6315°N, 120.6182°W) represents the most productive and historically reliable overwintering site on VSFB. Located in South Base within 300 meters of Space Launch Complex 4, this approximately 2.0-hectare site consisted entirely of mature blue gum eucalyptus trees reaching heights of approximately 40 meters. An unnamed perennial creek runs through the center of the grove, creating a riparian corridor that supports heterogeneous canopy structure with variable tree spacing and diverse understory vegetation. Surf Road, a paved vehicle access road used infrequently, bisected both the perennial creek and forest canopy. 

The UDMH site (34.6719°N, 120.5950°W), also located in South Base, comprises a 5.1-hectare eucalyptus grove planted in windrows adjacent to a waste treatment facility. The uniformly spaced trees maintain a largely clear understory with scattered low shrubs. Although only recently documented as an overwintering location in 2022, UDMH immediately emerged as a significant site, supporting over 6,000 monarchs during its initial count and ranking among the base's highest population sites.

% [Figure here showing all thirteen groves, with different symbology for the various subsets]

\subsection{Monitoring Strategy}

Equipment deployment strategies differed between monitoring seasons to accommodate research objectives and field experience. During the 2023-2024 season, equipment deployment followed two strategies: (1) targeted deployments at sites with confirmed monarch presence, and (2) anticipatory deployments at locations where monarchs were expected based on historical data but not currently observed. Targeted deployments were concentrated at Spring Canyon and UDMH where active aggregations were documented throughout the season. Anticipatory deployments were made at four overwintering sites: additional locations within Spring Canyon and UDMH, plus SLC-6 and Tangair. No monarchs were recorded at any anticipatory deployment sites, and these data are not included in this study.

For the 2024-2025 season, we modified our approach to establish monitoring stations at ten sites before monarch arrival, based on historical occurrence records compiled by the base conservation coordinator. This expanded spatial coverage was designed to capture greater environmental variation across potential overwintering sites not captured in our 2023 season. However, the 2024-2025 season coincided with historically low monarch abundance throughout California \autocite{xercesWesternMonarchThanksgiving2025}, resulting in no observed clustering behavior at any monitored location on base. Consequently, our final dataset was restricted to two sites, Spring Canyon and UDMH, during the 2023 season only.

\subsection{Field Equipment}

To observe changes in monarch abundance in response to strong wind events, we deployed remote monitoring equipment near monarch butterfly clusters at overwintering sites. Field observations were conducted using 15-meter telescoping fiberglass poles (Max-Gain Systems, Inc., Marietta, GA). Each pole was anchored at three points using ground anchors with guy lines securing both the top and base to create a stable, freestanding structure. 

Poles were positioned between 4 and 17 meters from cluster locations. This range, determined through field testing, balanced image resolution requirements for our grid-based counting method against disturbance minimization. Closer positioning compromised field of view, while greater distances degraded butterfly visibility below classification thresholds. Pole placement sites were selected based on ground stability for the 15-meter structures, infrastructure clearance requirements at the military installation, and clear viewing angles. When deploying near active clusters, we approached from directions that avoided direct disturbance. At no time did butterflies show signs of dispersal. 

We monitored monarch abundance using modified trail cameras (GardePro E7 and E8, Shenzhen, China) configured for near-infrared imaging to enhance contrast between clustering butterflies and surrounding vegetation. Trail cameras were selected for their durability in extended field deployment, native time-lapse functionality, and suitability for modification. Near-infrared wavelength selection was based on previous literature demonstrating potential effectiveness for butterfly population estimation \autocite{hristov_estimating_2019}. 

Hardware modifications exploited the camera's internal filter-switching mechanism by engaging the nighttime mode to access the clear glass filter position, then disconnecting power to prevent reversion to the infrared cut filter. Near-infrared pass filters (>850 nm) were mounted externally to restrict incoming light to NIR wavelengths only. This configuration produced images where clustering butterflies appeared as dark masses against the bright reflectance of eucalyptus foliage in the near-infrared spectrum. Field validation confirmed that modified cameras provided sufficient contrast for visual distinction of monarch clusters from background vegetation, supporting our human-labeler analytical approach.

Cameras were mounted at the top of poles using lightweight tie-down straps and positioned horizontally toward butterfly clusters at their roosting height. The wireless live view feature allowed for real-time preview and precise camera aiming during deployment. We configured cameras for time-lapse mode with motion detection disabled. 

Sampling interval selection was based on empirical optimization balancing temporal resolution, battery life, and data processing feasibility. Initial deployments used 10-minute intervals, selected to capture significant changes in butterfly abundance that preliminary observations showed occurred on hourly rather than minute scales, while maintaining approximately 6-week continuous operation under field conditions. Post-deployment statistical analysis using mixed-effects models and information-theoretic approaches demonstrated that 30-minute intervals provided optimal balance, losing less than 5\% of information compared to full temporal resolution (measured by root mean square error) while reducing image classification workload by 67\%. This interval preserved essential time-series patterns including diurnal activity cycles, weather-response dynamics, and multi-day population trends across deployments, as validated through both quantitative metrics and visual inspection of temporal patterns. Battery life constraints and field deployment logistics further supported this interval choice, allowing extended autonomous operation essential for capturing complete behavioral sequences during variable weather conditions.

Wind monitoring equipment consisted of Rain Wise WindLog Wind Data Loggers (Rain Wise Inc., Trenton, Maine) installed at the apex of each pole. These instruments recorded average wind speed and maximum wind gust at one-minute intervals. This configuration allowed wind measurements at heights approximating butterfly roosting locations, providing data on microclimate conditions that may influence cluster dynamics.

\subsubsection{Strong Wind Event Definition}

Strong wind events were defined using the established Kingston Threshold of 5 mph (2.2 m/s) \autocite{kingston_effect_2007}, which represents the canonical threshold considered by the broader monarch research community prior to this study. This threshold served as our \textit{a priori} hypothesis for the minimum wind speed expected to elicit behavioral responses in overwintering monarch butterflies. While our subsequent findings suggest the biologically relevant threshold may be higher, our methodological approach was grounded in this established baseline to maintain consistency with previous research and provide a testable hypothesis framework.

\subsubsection{Temporal Synchronization Methodology}

The temporal linking of wind measurements and photographic observations required careful consideration of sampling frequencies and data processing approaches. Photographic data were collected at regular intervals throughout each deployment period, while wind measurements were recorded continuously at one-minute intervals. This deliberate mismatch in sampling frequencies served multiple methodological purposes that enhanced our analytical capabilities beyond what uniform sampling intervals would have provided.

The higher frequency wind data collection enabled calculation of wind speed variance (standard deviation) within each photographic sampling period, providing a novel metric of wind gustiness that could not be captured through averaged measurements alone. This approach preserved maximum data integrity, as high-frequency measurements could be aggregated to match photographic intervals while retaining information about short-term wind variability that may influence butterfly behavior. Recording wind data at one-minute intervals represented the highest frequency supported by the sensors while creating no significant battery life trade-offs compared to longer intervals.

During analysis, wind metrics including average speed, maximum gust, and variance were calculated for time periods corresponding to each photographic sample, directly linking environmental conditions to observed changes in monarch abundance. This synchronization approach allowed us to examine both immediate responses to wind conditions and potential lagged effects across multiple temporal scales.

To systematically organize our heterogeneous monitoring efforts and create consistent analytical units, we defined discrete monitoring periods as deployment units. Each deployment represented a unique combination of monitoring location, camera configuration (including camera ID, mounting height, and viewing angle), associated wind measurements, and temporal coverage period. Since equipment was frequently reused across different locations and time periods, this deployment-based structure provided named units of sampling effort that allowed us to account for variation in environmental conditions and equipment configurations while treating each deployment as an independent sampling unit in subsequent statistical analyses. This approach produced time-series images from each deployment, which we used to estimate monarch cluster abundance over time through systematic grid-based counting methods. These abundance estimates could then be analyzed in relation to wind speed measurements and other environmental variables recorded at each deployment location.

% [A tree with a measuring device and a diagram AI-generated content may be incorrect.]

\subsection{Image Analysis}

\subsubsection{Grid-based Counting Method}

To quantify changes in monarch butterfly abundance from collected imagery, we developed a systematic grid-based counting protocol that balanced accuracy with the practical constraints of analyzing tens of thousands of images. This approach addressed the challenge of estimating abundance in large aggregations where individual counts would be prohibitively time consuming. This approach also emulates the manner in which abundance estimates are collected by field researchers, including those conducting the annual Thanksgiving Count \autocite{xercesProtocolsWesternMonarch2017}. We subdivided each image using a grid overlay system, where human labelers could assign order-of-magnitude estimates per cell. Grid dimensions remained fixed throughout each deployment to ensure consistency. To facilitate this labeling effort, we developed custom software using the Electron framework in JavaScript.

Grid cell size varied by deployment based on camera-to-cluster distance. Cell dimensions were optimized to ensure most occupied cells would contain butterflies in the 10-99 count range, balancing classification efficiency with spatial resolution. This standardization minimized the occurrence of cells alternating between widely different order-of-magnitude categories across the time series.

\subsubsection{Counting Protocol}

Human labelers estimated butterfly abundance within each grid cell using four order-of-magnitude categories: 0 (no butterflies), 1-9 (single digits), 10-99 (dozens), and 100-999 (hundreds). Labelers were trained using a comprehensive online guide that included example images for each category and detailed classification criteria (\url{https://kylenessen.github.io/monarch_trailcam_classifier/}). The protocol prioritized efficiency while maintaining consistency across observers.

Because abundance estimates were derived exclusively from two-dimensional photographic images, our classification protocol focused on quantifying only butterflies visible in the image plane without attempting to estimate the three-dimensional structure or depth of clusters. This approach intentionally excludes hidden individuals positioned behind visible butterflies in overlapping aggregations, providing a conservative but consistent measure of cluster size that reflects the observable surface area of each aggregation rather than its total volume. For cells containing partial butterflies at grid boundaries, labelers included these in counts unless double-counting would cause an adjacent cell to move to a higher category. When butterfly counts in a cell fluctuated between categories across the time series, the lower estimate was consistently applied to provide conservative abundance estimates.

In addition to estimating monarch abundance, labelers were asked to record if the cell was in direct sunlight or not. Direct sunlight classification presented challenges because the oversaturated conditions eliminated the contrast that made butterfly detection possible in shaded areas. Labelers were instructed to classify cells as receiving direct sunlight when branches or butterflies exhibited additional illumination that was clearly direct rather than indirect light, even when individual butterflies became difficult to distinguish due to pixel oversaturation. This classification required careful attention to subtle shape recognition and maintenance of contextual awareness about butterfly locations established from previous images in the time series. This measurement was recorded only if the cell was occupied and was stored as a separate value.

Labelers received ongoing feedback throughout the classification process. All classifications were reviewed for common errors including mislabeled cells, incorrect category assignments, and inconsistent application of counting criteria. We communicated corrections directly to labelers to ensure consistent application of the protocol.

\subsubsection{Abundance Calculation}

We calculated an abundance index for each frame by summing the products of cell counts and their assigned category values across all grid cells. This index employed conservative estimates using the minimum value within each order-of-magnitude category:

\begin{equation}
\text{Abundance index} = \sum_{i} \rho_i \times C_i
\end{equation}

where $\rho_i$ represents the number of cells in category $i$, and $C_i$ represents the conservative estimate for that category. We used minimum category values ($C_1 = 1$ for category 1-9, $C_2 = 10$ for category 10-99, and $C_3 = 100$ for category 100-999) rather than midpoint or maximum values to ensure our temporal analyses reflected genuine population shifts in categories rather than estimation uncertainty.