\section{Discussion}

\subsection{Wind Does Not Disrupt Overwintering Monarch Butterflies}

Our study provides the first direct empirical test of the disruptive wind hypothesis and finds no support for wind as a primary factor influencing monarch butterfly clustering behavior. Despite the widespread adoption of the 2 m/s wind threshold in conservation practice \autocite{xercesGuideWesternMonarch2025}, our data reveal no relationship between wind speed and butterfly departures across the full range of observed conditions (0--12 m/s). While our models explain only 5.7\% of variance in butterfly movements, reflecting our focus on testing wind effects rather than comprehensively explaining movement patterns, they had sufficient statistical power to detect environmental signals. This finding challenges assumptions underlying three decades of management guidance and suggests that our understanding of overwintering habitat requirements warrants reconsideration.

The absence of wind effects in our data is particularly striking given that mean maximum wind speeds (2.2 m/s, SD = 1.4) frequently exceeded the proposed threshold. If the disruptive wind hypothesis were valid, we should have observed a clear signal: transitions from aggregated butterflies to zero butterflies, as strictly predicted by the hypothesis. Instead, we observed no change, small changes, or even positive changes in butterfly abundance at wind speeds six times the proposed threshold. 

Importantly, our analysis had adequate statistical power to detect biologically meaningful wind effects. Power analysis demonstrated 87.5\% power to detect moderate effect sizes (0.15 standard deviations) and 98.5\% power for larger effects (0.20 standard deviations). Among our 48 candidate models, wind appeared in only one of the top five models (M24), where it showed little evidence of an effect (p = 0.218) and resulted in substantially poorer model performance compared to the best model (ΔAIC = 6.2, capturing only 4\% of model weight). This weak wind signal, combined with our high statistical power, allows us to rule out all but very small wind effects. Given that the disruptive wind hypothesis predicts complete cluster abandonment above threshold speeds, a large effect by any measure, our failure to detect such patterns provides strong evidence against the hypothesis rather than merely absence of evidence.

The methodological validity of our approach is confirmed by the strong signals detected for other environmental variables. Had our counting method or analytical framework been flawed, we would not have captured the pronounced effects of direct sunlight (F = 19.36, p < 0.001) or the complex diurnal patterns (F = 8.90, p < 0.001) that emerged from the same dataset.

\subsection{Alternative Drivers of Monarch Movement}

While our study was designed specifically to test the wind hypothesis rather than identify all factors influencing monarch movement, our results suggest that thermoregulation, light exposure, and diurnal rhythms may play more important roles than wind in driving short-term movements at overwintering sites. These patterns align with fundamental principles of insect physiology and behavior, though further research is needed to definitively establish their relative importance.

\subsubsection{Direct Sunlight as the Strongest Predictor}

Direct sunlight exposure emerged as the strongest environmental predictor of cluster abandonment in our study (F = 19.36, p < 0.001). Butterflies exposed to direct sunlight at the beginning of an observation interval showed the largest decreases in abundance, suggesting that solar radiation rapidly warms butterfly body temperatures well above ambient conditions. This finding aligns with \citeauthor{masters_thermal_1988}'s work showing that monarchs in direct sunlight can elevate their body temperature above ambient conditions within minutes \autocite{masters_thermal_1988}. This rapid warming capability explains why direct sunlight exposure is such a strong predictor of cluster abandonment, butterflies must quickly respond to avoid overheating.

The relationship between sunlight and departure represents a key component of the thermoregulatory equation. While monarchs have evolved efficient solar heat absorption capabilities that enable activity during cool conditions, this same efficiency forces them to abandon energetically favorable clustering positions when exposed to direct solar radiation. This trade-off between the benefits of clustering and the thermal constraints imposed by solar exposure may fundamentally shape daily movement patterns at overwintering sites.

\subsubsection{Temperature Effects and Their Interpretation}

Ambient temperature showed a more subtle relationship with monarch abundance changes (EDF = 3.93, F = 3.23, p = 0.028). While this effect was statistically significant, the confidence intervals contained zero throughout much of the temperature range, requiring cautious interpretation. The data suggest minimal change in clustering behavior below 15°C, which aligns with the known flight threshold for monarchs. Between 15°C and 25°C, there appears to be a slight positive association with cluster formation, peaking around 20--21°C. This pattern might indicate that monarchs preferentially select cluster sites within this temperature range, though our single-year, single-site study cannot conclude this definitively.

Above 25°C, the data show a sharp decline in cluster size, consistent with thermoregulatory constraints. However, given that thermal preferences can vary widely between sites as demonstrated by \citeauthor{SanieeThesis2016} \autocite{SanieeThesis2016}, we should be cautious about over-extending generalizations from our study. The large confidence intervals around temperature effects reflect both the inherent variability in butterfly responses and the limitations of our counting methodology.

\subsubsection{Diurnal Activity Patterns}

Time since sunrise revealed distinct patterns in butterfly movements (EDF = 4.90, F = 8.90, p < 0.001), though confidence intervals contained zero throughout much of the range. Despite this statistical limitation, the data show a clear trend: butterflies tend to depart clusters in the morning and reform aggregations in the afternoon. This pattern aligns with numerous anecdotal observations from overwintering sites and suggests an underlying diurnal rhythm that persists even after controlling for temperature and sunlight.

The detection of this pattern despite large confidence intervals is noteworthy. The variability likely reflects both the limitations of our counting methodology and the complex nature of butterfly behavioral responses. That we can detect this diurnal signal with all other factors held equal suggests it represents a genuine biological pattern, though the magnitude of the effect remains uncertain.
\subsection{Light as a Predictable Environmental Cue}

The strong effect of direct sunlight on butterfly movements (F = 19.36, p < 0.001) suggests light exposure may be a key driver of clustering behavior. Unlike wind, which varies stochastically at fine spatial scales, light patterns created by canopy structure remain relatively stable across seasons and years. This stability could provide monarchs with reliable cues for habitat selection, potentially explaining the consistent use of specific trees and branches documented across overwintering sites. The relationship between light exposure and thermoregulation, combined with the predictability of canopy-created light patterns, warrants further investigation as a primary factor in roost site selection. % This sound a lot like future research ideas, which it should be included in or canabilized for that section for content.

\subsection{Study Limitations}

Several limitations of our study warrant consideration when interpreting these findings. First, our data derive from a single overwintering season (2023--2024) at only a few monitoring locations, primarily within Spring Canyon grove, with additional limited data from UDMH. The historically low monarch abundance during our second field season (2024--2025) prevented temporal replication. Second, monarch populations during our study period were substantially lower than during the early 1990s when the wind hypothesis was first formulated. It remains possible that wind effects manifest differently at higher butterfly densities.

Third, our counting methodology, while necessary to process tens of thousands of images, introduced discretization artifacts and reduced temporal resolution. The discrete nature of our counts (0, 1--10, 11--100, etc.) contributes to the large confidence intervals observed for our environmental predictors. This methodological limitation means that while we can detect strong signals like direct sunlight effects, more subtle relationships require careful interpretation.

\subsection{Management Implications}

Our findings suggest that management strategies prioritizing wind protection warrant reconsideration. While our study represents a single season at one primary site, the absence of wind effects despite frequent threshold exceedances raises questions about current habitat assessment criteria.

If wind is not a primary limiting factor, the usable habitat within existing groves may be much larger than currently recognized. Conservation practice often assumes that only a small portion of each grove provides suitable conditions, with the remainder serving primarily as a wind buffer. However, areas within groves previously dismissed due to perceived wind exposure may actually provide suitable overwintering conditions if they offer appropriate light regimes and thermal conditions. The historically low monarch numbers over the past decade may have concentrated butterflies in traditionally favored locations, leaving much of the potentially suitable habitat within each grove unoccupied. %Therefore, it is possible that existing overwintering sites maybe able to hold more butterflies than previously realized, which should encourage land managers and conservationists. Still it is iimportant to maintain these existing groves.

While past management efforts aimed at wind protection may have been based on incomplete understanding, they likely produced beneficial outcomes by increasing tree density through outplanting. The fundamental recommendation to plant and maintain trees remains sound, even as our understanding of the underlying mechanisms continues to evolve.

The widespread use of large trees by monarchs across overwintering sites supports continued emphasis on tree conservation and planting. Management should prioritize maintaining existing mature trees while establishing future roosting habitat. We recommend planting trees at the highest densities that support healthy, long-lived growth, as trees can be selectively thinned if needed but cannot be quickly replaced once lost.

\subsection{Future Research Directions}

Our findings open several important avenues for future research. First, explicit testing of light patterns as predictors of clustering locations could establish whether canopy structure guides habitat selection. Given the strong effect of direct sunlight in our models, understanding how monarchs use visual cues to evaluate potential roost sites could provide predictive power for identifying suitable habitat.

Second, investigation of social dynamics and positive feedback mechanisms could explain some of the unexplained variation in our models. Monarchs may exhibit emergent clustering behaviors where initial settlement by a few individuals increases the probability of others joining, creating self-reinforcing patterns independent of immediate environmental conditions. Understanding whether clustering exhibits emergent properties beyond individual decision-making could substantially improve our understanding of cluster formation dynamics, informing our approach to habitat creation and management.

Research should also examine whether our findings extend across the broader overwintering range. Testing these patterns at sites with different tree species, latitudes, and population densities would strengthen conclusions about the generality of wind effects, or their absence. Long-term studies capturing year-to-year variation in the same groves could distinguish between stochastic and deterministic processes in site selection.

\subsection{Conclusions}

The monarch butterfly's overwintering phenomenon represents one of nature's most remarkable migrations, transforming coastal California groves into spectacular aggregations each winter. Within our study system, wind did not disrupt monarch clusters even at speeds far exceeding established thresholds. Instead, our data suggest butterflies responded more strongly to thermal conditions, light exposure, and diurnal rhythms, though the relative importance of these factors requires further investigation.

These findings suggest the need to reevaluate assumptions about overwintering habitat requirements. While our single-season study at one primary grove cannot definitively refute decades of observation across the entire overwintering range, the absence of wind effects in our data raises important questions about current management priorities. The conservation community should remain open to updating practices as new empirical evidence emerges. As monarch populations face continued threats from habitat loss and climate change, evidence-based management becomes increasingly critical. By better understanding the factors driving habitat selection and clustering behavior, we can more effectively conserve the overwintering sites essential for the persistence of this iconic species.