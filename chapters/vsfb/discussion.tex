\section{Discussion}

\subsection{Wind Does Not Disrupt Overwintering Monarch Butterflies}

Our study provides the first direct empirical test of the disruptive wind hypothesis and finds no support for wind as a primary factor influencing monarch butterfly clustering behavior. Despite the widespread adoption of the 2 m/s wind threshold in conservation practice \autocite{xercesGuideWesternMonarch2025}, our data reveal no relationship between wind speed and butterfly departures across the full range of observed conditions (0--12 m/s). This finding challenges three decades of management guidance based on indirect evidence and fundamentally questions our understanding of overwintering habitat requirements.

The absence of wind effects in our data is particularly striking given that mean maximum wind speeds (2.2 m/s, SD = 1.4) frequently exceeded the proposed threshold. If the disruptive wind hypothesis were valid, we should have observed a clear signal: transitions from aggregated butterflies to zero butterflies, as strictly predicted by the hypothesis. Instead, we observed no change, small changes, or even positive changes in butterfly abundance at wind speeds six times the proposed threshold. The methodological validity of our approach is confirmed by the strong signals detected for other environmental variables. Had our counting method or analytical framework been flawed, we would not have captured the pronounced effects of direct sunlight (F = 19.36, p < 0.001) or the complex diurnal patterns (F = 8.90, p < 0.001) that emerged from the same dataset.

\subsection{Alternative Drivers of Monarch Movement}

While our study was designed specifically to test the wind hypothesis rather than identify all factors influencing monarch movement, our results suggest that thermoregulation, light exposure, and diurnal rhythms may play more important roles than wind in driving short-term movements at overwintering sites. These patterns align with fundamental principles of insect physiology and behavior, though further research is needed to definitively establish their relative importance.

\subsubsection{Thermoregulation as a Potential Driver}

Temperature effects on monarch abundance changes exhibited a clear optimum at approximately 20�C, consistent with previous work on thermal requirements of overwintering monarchs \autocite{masters_thermal_1988}. The non-linear relationship we observed suggests monarchs actively manage their body temperature through movement between microhabitats. Below the thermal optimum, butterflies remained clustered, likely conserving energy through reduced metabolic rates. Above 20�C, increased departures may reflect butterflies seeking cooler microsites or engaging in flight-dependent activities enabled by warmer conditions.

The strong negative effect of direct sunlight on roost abundance provides critical insight into thermoregulatory behavior. Butterflies exposed to direct sunlight at the beginning of an observation interval showed the largest decreases in abundance, suggesting rapid warming triggers immediate behavioral responses. This finding aligns with studies showing that even 10 minutes of direct sun exposure can raise thoracic temperatures to levels requiring behavioral modification \autocite{masters_thermal_1988}. The relationship between sunlight and departure helps resolve an apparent paradox: while monarchs have evolved efficient solar heat absorption capabilities, excessive warming forces them to abandon energetically favorable clustering positions.

\subsubsection{Diurnal Activity Patterns}

The pronounced diurnal pattern in butterfly movements persisted even after controlling for temperature and sunlight, indicating an underlying circadian rhythm. Butterfly numbers typically increased during morning hours, then decreased as temperatures warmed, with most individuals departing by midday. This pattern suggests that rapid warming capability through solar absorption enables monarchs to engage in daily activities while returning to roosts as conditions cool. The ability to quickly elevate body temperature through basking may thus facilitate, rather than disrupt, the maintenance of overwintering aggregations by allowing brief periods of activity within an otherwise quiescent state.

\subsubsection{Stochastic Clustering Processes}

The relatively low explanatory power of our best model (R² = 0.057) suggests that factors beyond those measured may influence short-term clustering dynamics. One possibility is that monarch aggregation sites function through positive feedback mechanisms, where initial settlement by a few individuals increases the probability of additional butterflies joining that location. This hypothetical "seeding" behavior could create self-reinforcing clustering patterns independent of immediate environmental conditions. Such emergent social dynamics, if they exist, would make short-term movements difficult to predict from abiotic variables alone, while still producing the consistent year-to-year use of specific roost sites observed throughout the overwintering range.

\subsection{The Predictability of Light Versus Wind}

The dominance of light-related variables in our models points to a fundamental difference in the predictability of environmental cues. Light patterns created by canopy structure remain stable throughout seasons and across years, providing reliable indicators of microhabitat quality. A monarch's compound eye, with its numerous ommatidia, appears well-suited to assess canopy cover and light availability across potential roost sites. This visual assessment capability could explain the remarkable fidelity to specific trees and even individual branches documented across overwintering sites.

In contrast, wind protection can only be evaluated during actual wind events. The stochastic nature of wind exposure at fine spatial scales makes it an unreliable cue for habitat selection. While catastrophic wind events may occasionally impact overwintering populations, our data suggest that monarchs have not evolved specific behavioral responses to moderate wind speeds. Instead, selection appears to have favored responses to predictable environmental gradients like light and temperature that consistently indicate suitable microclimates.

\subsection{Study Limitations}

Several limitations of our study warrant consideration when interpreting these findings. First, our data derive from a single overwintering season (2023--2024) at only a few monitoring locations, primarily within Spring Canyon grove, with additional limited data from UDMH. The historically low monarch abundance during our second field season (2024--2025) prevented temporal replication. Second, monarch populations during our study period were substantially lower than during the early 1990s when the wind hypothesis was first formulated. It remains possible that wind effects manifest differently at higher butterfly densities or that contemporary populations represent individuals selected for wind tolerance.

Third, our counting methodology, while necessary to process tens of thousands of images, introduced discretization artifacts and reduced temporal resolution. The 30-minute sampling interval may have missed brief wind-induced disturbances that quickly resolved. However, any such transient effects would need to be reconciled with the hypothesis's prediction of sustained abandonment above the wind threshold.

Finally, the low variance explained by our best model indicates substantial unexplained variation in butterfly movements. However, this limitation does not diminish our primary conclusion. The study was explicitly designed to test the wind hypothesis rather than comprehensively explain monarch movement patterns. The absence of wind effects in our models, despite detecting clear signals for other variables, provides strong evidence against wind as a primary driver of clustering behavior.

\subsection{Management Implications}

Our findings suggest that current management strategies prioritizing wind protection may be misdirected. Three decades of habitat management based on the 2 m/s threshold have likely diverted resources from more critical habitat attributes. However, this realization also presents opportunities for innovative conservation approaches.

The absence of wind constraints implies more potential habitat exists within current groves than previously recognized. Sites dismissed due to perceived wind exposure may actually provide suitable overwintering conditions if other key factors, particularly appropriate light regimes and thermal conditions, are present. The historically low monarch numbers over the past decade may have masked the suitability of these alternative sites by concentrating the remaining butterflies in traditionally favored locations. This expanded habitat potential could inform restoration efforts, including strategic canopy management to create favorable light patterns. Indeed, Kingston Leong's experimental canopy modifications in Los Osos appear to have successfully created monarch habitat \autocite{kingston_spatial_2016}, suggesting that active management based on factors other than wind can be effective.

While our study does not directly address tree size preferences, the widespread use of large trees by monarchs across overwintering sites supports continued emphasis on mature tree conservation. Management strategies should prioritize the health of existing mature trees while establishing the next generation of roosting habitat. We recommend planting trees at the highest density that supports healthy, long-lived growth. Trees can be selectively removed as our understanding improves, but they cannot be quickly grown to replace losses.

Interestingly, while past management efforts aimed at wind protection may have been based on incorrect assumptions, they produced beneficial outcomes by increasing tree density and canopy complexity. The recommendation to plant and maintain trees remains valid, even as our understanding of the underlying mechanisms shifts from wind protection to light and thermal regulation.

\subsection{Future Research Directions}

Our findings open several important avenues for future research. First, explicit testing of light patterns as predictors of clustering locations could establish whether visual assessment of canopy structure guides habitat selection. Second, investigation of social dynamics and positive feedback mechanisms in cluster formation could explain the stochastic patterns we observed. Understanding whether clustering exhibits emergent properties beyond individual decision-making could substantially improve our approach to habitat creation and management.

Research should also examine whether our findings extend across the broader overwintering range. Testing these patterns at sites with different tree species, latitudes, and population densities would strengthen conclusions about the generality of wind effects, or their absence. Long-term studies capturing year-to-year variation in the same groves could distinguish between stochastic and deterministic processes in site selection.

Finally, mechanistic studies of monarch sensory capabilities and behavioral algorithms for habitat assessment could bridge the gap between environmental conditions and clustering behavior. Understanding how monarchs perceive and evaluate potential roost sites would provide predictive power for identifying and creating suitable habitat as climate change shifts overwintering distributions.

\subsection{Conclusions}

The monarch butterfly's overwintering phenomenon represents one of nature's most remarkable migrations, transforming coastal California groves into spectacular aggregations each winter. Our study demonstrates that wind, long considered a primary limiting factor for overwintering habitat, does not disrupt monarch clusters even at speeds far exceeding established thresholds. Instead, our data suggest butterflies may respond more strongly to thermal conditions, light exposure, and diurnal rhythms, though the relative importance of these factors requires further investigation.

These findings mandate a fundamental reconsideration of overwintering habitat management. While uncertainty remains about many aspects of monarch clustering behavior, our results clearly indicate that management should not prioritize wind protection. The conservation community must be willing to update practices as empirical evidence supersedes indirect inference, even when challenging long-held assumptions. As monarch populations face continued threats from habitat loss and climate change, evidence-based management becomes increasingly critical. By understanding the actual factors driving habitat selection and clustering behavior, we can more effectively conserve the overwintering sites essential for the persistence of this iconic species.