\section{Discussion}

Our study provides the first direct empirical test of the long-standing hypothesis that wind disrupts overwintering monarch butterfly clusters. For over three decades, conservation practice has operated under the assumption that wind speeds exceeding 2 m/s force butterflies to abandon their roosts, either by physically dislodging them or triggering behavioral departures \parencite{leongEvaluationManagementCalifornia2016}. Our findings challenge this fundamental assumption and suggest a more complex relationship between monarchs and their overwintering environment.

\subsection{Evidence Against Wind Disruption}

The evidence against the wind disruption hypothesis emerges from multiple, independent lines of analysis. Most strikingly, every single observation period in our day-to-day analysis experienced maximum wind speeds exceeding the proposed 2 m/s threshold (range: 2.0--12.8 m/s), yet monarch clusters persisted throughout the 78-day study period. If the hypothesis were correct, we should have observed either mass departures when temperatures permitted flight or butterflies physically dislodged and littering the ground when too cold to fly, as described in the literature \parencite{leongRestorationOverwinteringGrove1999}. We observed neither.

Our bivariate analyses provide additional refutation. When we examined the simple relationship between wind speed and cluster size changes, we found essentially no correlation at either temporal scale (30-minute r = 0.04, n = 1,894; day-to-day r = 0.13, n = 96). These analyses tested the most basic prediction of the hypothesis that increasing wind speed should produce decreasing cluster sizes. The absence of this relationship across wind speeds ranging from calm conditions to six times the proposed threshold suggests that wind alone does not drive clustering decisions.

Statistical power was not a limitation. Our analysis achieved 87.5\% power to detect moderate effects and 98.5\% power for large effects. The wind disruption hypothesis predicts substantial, observable impacts, not subtle statistical signals. Our failure to detect these effects, despite adequate power and validated methodology that successfully identified other environmental signals, provides strong evidence against the hypothesis rather than merely absence of evidence.

\subsection{Thermoregulation as an Alternative Explanation}

While wind alone showed no disruptive effect, our models revealed that monarchs respond to environmental conditions through complex interactions, particularly between wind and solar exposure. This interaction emerged as the dominant environmental signal in both temporal analyses (30-minute F = 4.67, p < 0.001; day-to-day F = 4.10, p < 0.001), suggesting that the relationship between wind and monarch behavior depends critically on light conditions.

The interpretation of these wind-light interactions requires careful consideration. During our study, butterflies rarely experienced direct sunlight, with most observations recording few or no butterflies in sun. This natural behavioral pattern meant that our models extrapolated interaction effects into data-sparse regions, particularly at high sun exposure values where fewer than 3\% of observations occurred. The patterns we describe therefore reflect a combination of robust findings in data-rich conditions (primarily shaded butterflies) and model predictions in less-observed states.

The clearest finding emerges from our most common condition: when butterflies remained in shade, wind speed had no effect on cluster dynamics across the entire observed range (0 to 12.4 m/s). This pattern was consistent across both our 30-minute and sunset analyses, with only the most extreme observations in the sunset analysis showing any deviation, directly contradicts the wind disruption hypothesis. The interaction patterns at higher sun exposure, while statistically significant, derive largely from model interpolation. Intriguingly, the 30-minute and sunset analyses suggested different directional effects under similar wind-sun combinations, potentially reflecting distinct behavioral responses operating at immediate versus day-to-day temporal scales.

The mechanisms underlying these patterns remain uncertain given the sparse data at high sun exposure values. Thermoregulation offers one plausible interpretation, as \textcite{mastersMonarchButterflyDanaus1988} demonstrated that monarchs in direct sunlight can elevate body temperature above ambient conditions within minutes, potentially driving departures in calm conditions. The model-predicted patterns at intermediate wind and sun levels might reflect complex behavioral responses to multiple environmental cues. However, with fewer than 3\% of observations occurring at high sun exposure, these interaction patterns largely emerge from statistical extrapolation rather than direct observation. The contrasting patterns between our immediate (30-minute) and daily (sunset) temporal scales further suggest that multiple behavioral mechanisms may operate simultaneously. Rather than attempting to resolve these mechanisms with limited data from rare conditions, our findings point toward the critical importance of light exposure itself as a driver of butterfly behavior, warranting focused investigation of how canopy structure and the resulting light regimes influence cluster dynamics.

Additional observations align with thermal management. Time since sunrise revealed strong diurnal patterns (30-minute F = 9.85, p < 0.001), with butterflies departing clusters in late morning and early afternoon, then reforming aggregations later in the day. This pattern persisted after controlling for temperature and sunlight, potentially reflecting endogenous circadian rhythms, though it also aligns with predictable daily thermal cycles. The midday dispersal may reflect routine activities such as patrolling flights or nectaring. Monarchs have been observed gliding during these dispersal periods, which could facilitate thermoregulatory cooling while accomplishing other essential behaviors. Our study could not follow individual butterflies once they departed cluster locations, so these activities remain speculative. Similar temporal patterns have been consistently documented at California overwintering sites \parencite{tuskesOverwinteringEcologyMonarch1978,chaplinEnergyReservesMetabolic1982}, and the Xerces Society's standardized monitoring protocols explicitly restrict counting to specific time windows to account for these diurnal movements \parencite{xercessocietyStepbyStepWesternMonarch2017}.

Our models explained 6.4\% of variance in 30-minute cluster changes and 39.7\% of variance in day-to-day roost fidelity. The environmental factors that did explain variance were predominantly thermal in nature: the wind-light interaction, direct sunlight exposure, and diurnal patterns of dispersal and aggregation. While substantial variation remains unexplained, the environmental signals we detected suggest thermal factors may play a more important role than wind avoidance in organizing clustering behavior. Without direct body temperature measurements under varying environmental conditions, thermoregulation remains one plausible interpretation among potentially others, and further research is needed to confirm the underlying mechanisms.

\subsection{Limitations and Context}

Several factors shape the interpretation of our findings. First, our data come from a single overwintering season (2023--2024) when monarch populations were relatively typical \parencite{xercessocietyWesternMonarchThanksgiving2025}. The following season saw near-complete absence of monarchs at our study sites, coinciding with the second-lowest overwintering population on record \parencite{xercessocietyWesternMonarchButterfly2025}. This dramatic population crash prevented temporal replication but underscores the urgency of understanding overwintering ecology with the data we have.

Second, our study observed relatively small clusters where butterflies maintained direct contact with eucalyptus substrates. The wind disruption hypothesis was developed during an era of massive aggregations containing hundreds of thousands of individuals, where many butterflies attached only to other butterflies in multi-layered formations \parencite{leongMicroenvironmentalFactorsAssociated1990,browerMonarchButterflyClusters2008}. If substrate attachment provides greater wind resistance than butterfly-to-butterfly attachment, wind might affect these different clustering configurations differently. The hypothesis may have been accurate for the extreme densities of past decades but less relevant to today's smaller populations.

Finally, our models explained relatively little variance overall (30-minute 6.4\%; day-to-day 39.7\%), reflecting both the complexity of butterfly behavior and our focus on testing specific hypotheses rather than comprehensively explaining movement patterns. However, the strong signals we did detect and our adequate statistical power give confidence in our main conclusion that wind alone does not disrupt monarch clusters as previously believed.

\subsection{Implications for Conservation}

Our findings suggest that wind protection, long considered a limiting environmental factor in overwintering monarch conservation, may constrain habitat suitability far less than previously believed. This realization carries important implications for how we understand, manage, and restore overwintering sites.

\subsubsection{Expanded Habitat Availability and Population Limitations}

The absence of wind disruption despite frequent threshold exceedances indicates that suitable habitat within existing groves may be more extensive than currently recognized. Areas previously dismissed due to wind exposure might support clusters if they provide appropriate thermal and light conditions. With western monarch populations at historically low levels, this raises an important question: are we observing limited suitable habitat, or limited butterflies to occupy potentially suitable space? At historical abundances, monarchs may have utilized far more locations within groves than currently expressed. The small clusters we documented may represent only a fraction of available habitat, with suitable but unoccupied areas going unrecognized simply because too few butterflies exist to reveal these possibilities.

\subsubsection{Simplified Management Requirements}

Previous management frameworks required protecting extensive buffer zones around clustering sites. For example, \textcite{althouse&meadeinc.EllwoodMesaSperling2023} recommended maintaining trees up to six tree heights from clustering locations to ensure adequate wind protection. For mature eucalyptus reaching 40 meters, this translates to protecting all trees within approximately 240 meters, creating substantial complexity in coastal California where overwintering sites often involve diverse stakeholders and contentious land use decisions.

If wind protection plays a less critical role than assumed, these extensive buffer requirements may no longer be necessary. Trees distant from core groves, previously considered essential for wind breaks, may contribute far less than previously thought. This simplification focuses conservation efforts more precisely on immediate grove structure rather than attempting to control wind patterns across vast surrounding areas.

However, a precautionary approach remains warranted. Where buffer trees exist as healthy, mature specimens, maintaining them represents prudent hedging against incomplete understanding. This temporal asymmetry, where removing trees or modifying canopy structure is always possible while accelerating large tree growth is not, argues for maintaining existing trees wherever possible. Previous recommendations to plant trees, maintain them for longevity, and develop succession plans remain entirely valid, even if the mechanistic understanding of why these practices work requires updating.

\subsubsection{Proactive Habitat Creation Through Canopy Management}

If monarchs respond primarily to light and thermal patterns rather than strict wind thresholds, strategic canopy modification may create new overwintering habitat. The wind protection framework offered little beyond planting trees and waiting decades for maturation. Light-based management might enable shorter-timeframe interventions.

Real-world examples support this possibility. At Monarch Lane in Los Osos, strategic canopy opening resulted in monarchs appearing and persisting \parencite{leongRestorationOverwinteringGrove1999,xercessocietyWesternMonarchThanksgiving2025}. At Andrew Molera State Park in Big Sur, clearing an overgrown grove increased butterfly counts from zero to over one thousand (E. Pelton, personal communication). Conversely, tree planting at Pacific Grove required approximately 15 years of canopy maturation before monarchs returned (S. Weiss, personal communication). These patterns suggest that appropriate canopy structure, including carefully designed openings creating dappled light, can attract butterflies far more quickly than waiting for tree growth alone.

Unidentified eucalyptus stands or overly dense groves might thus be converted to functional habitat through selective thinning. However, additional research is needed before implementing large-scale canopy modifications, particularly at established overwintering sites. Rather than passive waiting, managers might proactively create suitable conditions by strategically opening existing canopies at appropriate locations, representing a significant departure from previous approaches focused almost entirely on wind protection and tree growth.

\subsubsection{Long-term Optimism Despite Current Uncertainty}

While this study reduces certainty regarding some aspects of overwintering ecology, the long-term trajectory points toward management practices that may be simpler to implement, grounded in empirical evidence, and potentially more effective. The strict 2 m/s threshold created an overly constraining understanding of habitat requirements, suggesting suitable sites were overly limited. Our results indicate monarchs may be more resilient to environmental variation than previously believed, potentially revealing more opportunities for habitat creation and maintenance.

The fundamental conservation tools remain unchanged: maintaining existing groves, planting and managing trees for longevity, and developing succession plans for aging forests. But we may now recognize more locations where habitat might be suitable and fewer absolute constraints on where sites can be maintained or created. As \textcite{sanieeHierarchyScaleInfluence2022} suggest, managing canopy structure to create appropriate light patterns and temperature gradients may prove more important than achieving specific wind speed thresholds. This perspective aligns with our finding that interactions between environmental factors, rather than single variables in isolation, shape clustering behavior.

Conservation efforts should continue protecting forest structure at overwintering sites while research clarifies the specific environmental factors, particularly light patterns and thermal regimes, that monarchs select for when choosing roost locations.

\subsection{Future Research Directions}

Our findings open several important avenues for future research. While the wind-light interaction we observed suggests complex relationships between environmental factors and clustering behavior, the strong effect of light exposure itself points toward canopy structure as a potentially primary driver of habitat selection. This conclusion finds support in \textcite{sanieeHierarchyScaleInfluence2022}, who found that while temperature and humidity failed to support the microclimate hypothesis at aggregation locations, solar radiation was one of the few environmental factors that distinguished occupied sites from other grove locations. Similarly, \textcite{weissForestCanopyStructure1991} found that successful overwintering sites maintain consistent canopy openness around 20\%, while unsuccessful sites show greater variability. Combined with our findings, this suggests that predictable light patterns created by canopy structure may provide a stable, reliable environmental cue that monarchs can consistently locate and respond to year after year. Unlike weather conditions that fluctuate unpredictably, the spatial pattern of light created by canopy architecture remains relatively constant across seasons, offering a dependable signal for site selection. Monarchs possess sophisticated visual systems that enable sun compass navigation during migration \parencite{nguyenSunCompassNeurons2021,mouritsenVirtualMigrationTethered2002}, capabilities that would allow them to detect and respond to consistent light patterns within groves. Future research should prioritize characterizing canopy structure and the resulting light regimes at both occupied and unoccupied sites to determine whether specific light patterns predict roost site selection and fidelity.

Beyond environmental factors, social dynamics may explain additional variation in our models. The strong effect of previous butterfly count on subsequent changes suggests that monarchs do not distribute randomly within groves but rather exhibit overdispersed clustering patterns where the presence of butterflies attracts others. This positive feedback mechanism, where initial settlement increases the probability of others joining, could create self-reinforcing aggregation patterns independent of immediate environmental conditions \parencite{berdahlEmergentSensingComplex2013}.

Finally, testing these patterns across the broader overwintering range would establish their generality. Sites with different tree species, latitudes, and especially population densities could reveal whether wind responses vary with clustering configuration or if our findings represent fundamental aspects of monarch overwintering behavior.

\subsection{Conclusions}

Our direct test of the wind disruption hypothesis found no evidence that wind speeds above 2 m/s force monarchs to abandon their clusters. Every observation in our day-to-day analysis exceeded this threshold, yet clusters persisted. Bivariate analyses showed no relationship between wind and cluster changes. Model selection consistently identified other factors as more important. These multiple lines of evidence converge on a clear conclusion. Wind alone does not disrupt overwintering monarch butterflies as has been assumed for over three decades.

Instead, our results suggest that thermal factors may play a more important role in clustering dynamics than previously recognized. Monarchs responded strongly to direct sunlight and diurnal patterns, both factors that could influence body temperature. The unexpected wind-light interaction, where moderate wind combined with sun exposure sometimes increased cluster sizes, suggests that environmental conditions interact in complex ways that simple threshold-based management approaches cannot capture.

These findings arrive at a critical moment for monarch conservation. With western populations at historic lows, every assumption about habitat requirements deserves scrutiny. While questions remain about the precise mechanisms, our findings demonstrate that current management guidelines based on wind speed thresholds are not supported by empirical evidence. Moving forward, conservation efforts should prioritize maintaining existing overwintering groves while additional research clarifies what environmental factors monarchs select for when choosing roost sites. As we face the challenge of preserving overwintering habitat for a declining population, evidence-based understanding of monarch ecology becomes not just scientifically important, but essential for preserving the remarkable phenomenon of monarch migration in western North America.