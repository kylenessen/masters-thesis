\section{Results}

\subsection{Descriptive Statistics}

Environmental conditions varied substantially across the 78-day monitoring period at Spring Canyon and UDMH during the 2023-2024 overwintering season. The dataset comprised 1,894 observations collected at 30-minute intervals during daylight hours (07:00--17:00) from November 17, 2023, to February 4, 2024, totaling 947 observation hours across 115 unique deployment-day combinations.

Wind speeds ranged from complete calm to moderately strong conditions, with maximum gusts reaching 12.4 m/s (mean = 2.2 ± 1.4 m/s, median = 2.2 m/s). The interquartile range of 1.3--3.0 m/s indicated that most observations occurred under relatively mild wind conditions. Temperature showed considerable variation throughout the monitoring period, ranging from 3.0 to 30.0°C (mean = 14.6 ± 3.8°C, median = 14.0°C), with an interquartile range of 12.5--17.0°C typical of California coastal winter conditions. Direct solar exposure occurred in 31.7\% of observations (n = 601), with butterflies actively basking when present in sunlight, averaging 17.0 individuals in direct sun (range: 1--295).

Monarch abundance exhibited high variability across sites and time periods. Butterfly counts ranged from 0 to 770 individuals per observation, with a mean of 81.4 ± 100.0 butterflies and a median of 37 butterflies. The wide interquartile range (9--119 butterflies) reflected substantial variation in cluster sizes. Zero-count observations, representing either the beginning of cluster formation or cluster dissolution, comprised 2.3\% of the dataset (n = 43).

Cluster sizes varied markedly among the 10 deployment locations. Site SC10 recorded the largest aggregation with 770 monarchs, while mean abundances ranged from 0 at SC9 to 325.8 at UDMH2. Eight deployments observed maximum cluster sizes exceeding 100 butterflies, with mean maximum cluster size across sites reaching 315.6 individuals. This variation in cluster sizes across deployments reflects the heterogeneous distribution of monarchs across overwintering microhabitats within the study sites.

The comprehensive temporal coverage, with observations at 30-minute intervals capturing 16.5 observations per deployment-day on average, provided fine-scale resolution of monarch behavioral responses to changing environmental conditions. Peak observation activity occurred at 16:00 hours (196 observations), corresponding with afternoon warming periods when monarchs typically exhibit increased movement.

\subsection{Responsive Change}

To examine how monarchs respond to immediate environmental conditions, we analyzed 1,894 paired observations collected at 30-minute intervals throughout the overwintering season. This responsive change analysis tested whether short-term fluctuations in cluster size could be explained by concurrent weather variables, particularly wind exposure.

\subsubsection{Model Selection}

We evaluated 52 candidate models using generalized additive mixed models (GAMMs) to identify the environmental predictors of monarch abundance changes. Model selection via Akaike Information Criterion (AIC) identified M50 as the decisively best-fit model, capturing 85.8\% of the total model weight (Table~\ref{tab:export-model-selection-table}). The next best model (M23) showed substantially weaker support with $\Delta$AIC = 3.8 and only 12.6\% model weight. Model M50 incorporated smooth terms for previous butterfly count, temperature, and time since sunrise, along with a tensor interaction between maximum wind speed and butterflies in direct sun.

\begin{table}

\caption{Top 5 models ranked by AIC (30-minute analysis)}
\centering
\begin{tabular}[t]{llrrrr}
\toprule
Model & Terms & AIC & Delta_AIC & Weight & Wind_p\\
\midrule
M50 & \textbullet\ Previous butterfly count\\ \textbullet\ Temperature\\ \textbullet\ Time since sunrise\\ \textbullet\ Interaction (tensor): Maximum wind speed, Butterflies in direct sun & 8074.029 & 0.000 & 0.8579 & 5.55e-05\\
M23 & \textbullet\ Previous butterfly count\\ \textbullet\ Temperature\\ \textbullet\ Butterflies in direct sun\\ \textbullet\ Time since sunrise & 8077.862 & 3.834 & 0.1262 & NA\\
M22 & \textbullet\ Previous butterfly count\\ \textbullet\ Temperature (linear)\\ \textbullet\ Butterflies in direct sun\\ \textbullet\ Time since sunrise & 8082.898 & 8.869 & 0.0102 & NA\\
M24 & \textbullet\ Previous butterfly count\\ \textbullet\ Maximum wind speed\\ \textbullet\ Temperature\\ \textbullet\ Butterflies in direct sun\\ \textbullet\ Time since sunrise & 8084.049 & 10.020 & 0.0057 & 2.18e-01\\
M52 & \textbullet\ Temperature\\ \textbullet\ Time since sunrise\\ \textbullet\ Interaction (tensor): Maximum wind speed, Butterflies in direct sun & 8092.722 & 18.693 & 0.0001 & 1.13e-05\\
\bottomrule
\end{tabular}
\end{table}
 

\subsubsection{Analysis of Best Fit Model}

The best-fit model (M50) predicted change in butterfly abundance as a function of: (1) the previous observation's butterfly count, (2) average temperature, (3) time since sunrise, and (4) an interaction between maximum wind gust speed and the number of butterflies in direct sunlight.

Statistical analysis of the model revealed significant effects for three of the four predictors (Table~\ref{tab:m50_summary}). The previous butterfly count (F = 12.50, p $<$ 0.001), time since sunrise (F = 9.85, p $<$ 0.001), and the wind-sunlight interaction (F = 4.67, p $<$ 0.001) all showed strong statistical significance. Temperature effects approached but did not reach conventional significance thresholds (F = 3.19, p = 0.057). The model explained 6.4\% of the variance in butterfly abundance changes (adjusted $R^2$ = 0.064, n = 1894).

\begin{table}[ht]
\centering
\caption{Summary of best-fit model (M50) for predicting changes in butterfly abundance. The model uses cube-root transformed butterfly count differences as the response variable.}
\label{tab:m50_summary}
\begin{tabular}{lcccc}
\toprule
\textbf{Smooth Term} & \textbf{edf} & \textbf{Ref.df} & \textbf{F} & \textbf{p-value} \\
\midrule
Previous butterfly count & 2.41 & 2.41 & 12.50 & $<$ 0.001*** \\
Average temperature & 3.68 & 3.68 & 3.19 & 0.057 \\
Time since sunrise & 4.87 & 4.87 & 9.85 & $<$ 0.001*** \\
Wind gust × Sunlight exposure & 7.35 & 7.35 & 4.67 & $<$ 0.001*** \\
\midrule
\multicolumn{5}{l}{\textit{Model performance:} Adj. $R^2$ = 0.064, Scale est. = 4.03, n = 1894} \\
\bottomrule
\end{tabular}
\end{table}

Partial effects for the smooth terms are shown in Figure~\ref{fig:partial_effects_30min}. The previous butterfly count showed a significant non-linear negative relationship (p < 0.001) consistent with proportionally greater departures from larger aggregations. Time since sunrise captured a significant diurnal pattern (p < 0.001) with morning departures and afternoon returns. Temperature showed a marginally non-significant trend (p = 0.057) suggesting possible effects near the 12.7–16°C flight threshold range.

\begin{figure}[htbp]
    \centering
    \includegraphics[width=\textwidth]{supplemental/results/30_min/figures/partial_effects_best_1x3.png}
    \caption{Partial effects of environmental predictors on monarch butterfly abundance changes from the best-fit GAMM model (M50). Shaded regions represent 95\% confidence intervals.}
    \label{fig:partial_effects_30min}
\end{figure}

The tensor interaction between wind speed and butterflies in direct sun (p < 0.001) revealed a conditional relationship where wind effects depended on solar exposure conditions (Figure~\ref{fig:interaction_wind_sun}). This interaction term improved model fit substantially, suggesting that the influence of wind on cluster dynamics varies with the thermal environment experienced by roosting monarchs.

\begin{figure}[htbp]
    \centering
    \includegraphics[width=0.8\textwidth]{supplemental/results/30_min/figures/interaction_wind_x_sun_binned.png}
    \caption{Interaction effect between maximum wind gust speed and butterflies in direct sun on abundance changes. The tensor smooth interaction reveals how wind effects vary with solar exposure conditions.}
    \label{fig:interaction_wind_sun}
\end{figure}

\subsubsection{Model Diagnostics}

Model diagnostic plots confirmed the adequacy of the GAMM specification (Figure~\ref{fig:model_diagnostics_30min}). Residual versus fitted value plots showed no systematic patterns or heteroscedasticity, indicating appropriate model structure. The quantile-quantile plot revealed approximately normal residual distribution with minor deviations in the tails, acceptable given the large sample size and complexity of ecological data. Basis dimension checks indicated adequate smoothing parameter selection, with all k-indices near 1.0 and p-values > 0.05, except for time since sunrise which showed marginal evidence of potential undersmoothing (k-index = 0.96, p = 0.065).

\begin{figure}[htbp]
    \centering
    \includegraphics[width=0.9\textwidth]{supplemental/results/30_min/figures/diag_qq_and_residuals_1x2.png}
    \caption{Diagnostic plots for the best-fit GAMM model (M50). Left panel shows quantile-quantile plot comparing model residuals to theoretical normal distribution. Right panel displays residuals versus fitted values to assess homoscedasticity and model adequacy.}
    \label{fig:model_diagnostics_30min}
\end{figure}

Temporal autocorrelation analysis revealed minimal residual correlation structure after accounting for the AR(1) correlation within deployment days (Figure~\ref{fig:acf_diagnostics}). The autocorrelation function showed rapid decay with all lags beyond lag 1 falling within the significance bounds, confirming that the model adequately captured temporal dependencies in the data. This indicates that our mixed-effects structure with autoregressive errors appropriately addressed the repeated measures nature of the time-series observations.

\begin{figure}[htbp]
    \centering
    \includegraphics[width=0.8\textwidth]{supplemental/results/30_min/figures/diag_acf.png}
    \caption{Autocorrelation function of model residuals showing minimal temporal correlation after accounting for AR(1) structure within deployment days. Blue dashed lines indicate 95\% confidence bounds for white noise.}
    \label{fig:acf_diagnostics}
\end{figure} 

\subsubsection{Sensitivity Analysis}

To test the hypothesis that wind acts as a binary disruption threshold rather than a continuous variable, we repeated the entire analysis using minutes above 2 m/s as the wind predictor. This threshold-based approach directly tests whether exceeding the proposed 2 m/s wind speed threshold drives butterfly movement, as suggested by previous literature. The best threshold model (T50) maintained the same structure as our primary analysis but showed notably weaker performance (Table~\ref{tab:exports}).

While T50 achieved the lowest AIC among threshold models, it failed to decisively outperform simpler alternatives, capturing only 55.4\% of model weight compared to 40.4\% for the model without any wind term (T23). The negligible difference in AIC ($\Delta$AIC = 0.63) between these models indicates substantial uncertainty about whether the threshold wind variable improves predictions. Furthermore, the wind threshold term showed weak statistical support even in the best model (p = 0.0001), and non-significant effects in alternative model specifications (e.g., T24: p = 0.372).

\begin{table}

\caption{\label{tab:exports}Model selection results from sensitivity analysis using wind threshold predictor (minutes with wind speed > 2 m/s). Poor model performance (high <U+0394>AIC values) demonstrates lack of support for the 2 m/s disruption threshold.}
\centering
\begin{tabular}[t]{llrrrl}
\toprule
Model ID & Terms included & AIC value & <U+0394>AIC & Weight & Wind threshold p-value\\
\midrule
T24 & - Previous butterfly count\textbackslash{}n- Minutes above 2 m/s\textbackslash{}n- Temperature\textbackslash{}n- Butterflies in direct sun\textbackslash{}n- Time since sunrise & 8089.408 & 7.560 & 0.0205 & 0.372\\
T44 & - Minutes above 2 m/s\textbackslash{}n- Temperature\textbackslash{}n- Butterflies in direct sun\textbackslash{}n- Time since sunrise & 8108.379 & 26.530 & 0.0000 & 0.256\\
T21 & - Previous butterfly count\textbackslash{}n- Minutes above 2 m/s\textbackslash{}n- Temperature\textbackslash{}n- Butterflies in direct sun & 8110.567 & 28.719 & 0.0000 & 0.0528\\
T43 & - Minutes above 2 m/s\textbackslash{}n- Temperature (linear)\textbackslash{}n- Butterflies in direct sun\textbackslash{}n- Time since sunrise & 8115.505 & 33.657 & 0.0000 & 0.275\\
T19 & - Previous butterfly count\textbackslash{}n- Minutes above 2 m/s\textbackslash{}n- Temperature (linear)\textbackslash{}n- Butterflies in direct sun & 8119.798 & 37.949 & 0.0000 & 0.0776\\
\bottomrule
\end{tabular}
\end{table}


The interaction plot from the threshold analysis provides additional evidence against meaningful wind effects on monarch clustering behavior (Figure~\ref{fig:threshold_interaction}). The relatively flat response surface across different combinations of wind exposure duration and solar conditions suggests that time above the 2 m/s threshold does not substantially influence butterfly abundance changes, regardless of thermal conditions. This lack of clear pattern, combined with the poor model performance, supports our conclusion that direct measurement of wind speed provides more appropriate characterization than binary threshold approaches, though neither reveals biologically important effects on monarch cluster dynamics.

\begin{figure}[htbp]
    \centering
    \includegraphics[width=0.8\textwidth]{supplemental/results/30_min_threshold/figures/interaction_wind_x_sun_binned.png}
    \caption{Interaction effect between minutes above 2 m/s wind threshold and butterflies in direct sun from sensitivity analysis. The minimal variation in response across wind exposure categories provides evidence against the threshold disruption hypothesis.}
    \label{fig:threshold_interaction}
\end{figure} 

\subsection{Statistical Power to Detect Wind Effects}

Post-hoc power analysis confirmed our study had adequate statistical power to detect biologically meaningful wind effects (Table~\ref{tab:power_analysis}). With 1,894 paired observations, we achieved 87.5\% power to detect moderate effect sizes (0.15 standard deviations) and 98.5\% power to detect larger effects (0.20 standard deviations). Power for small effects (0.10 standard deviations) was 56\%, while very small effects (0.05 standard deviations) yielded only 16.5\% power. These results indicate that our study has sufficient statistical power for effect sizes of biological relevance.

\begin{table}[htbp]
\centering
\caption[Statistical power to detect wind effects]{\textbf{Statistical power to detect wind effects.}}
\label{tab:power_analysis}
\begin{tabular}[t]{lrrl}
\toprule
  & Effect Size (SD units) & Power (Proportion) & Power (\%)\\
\midrule
0.05 & 0.05 & 0.165 & 16.5\%\\
0.1 & 0.10 & 0.560 & 56\%\\
0.15 & 0.15 & 0.875 & 87.5\%\\
0.2 & 0.20 & 0.985 & 98.5\%\\
\bottomrule
\end{tabular}
\end{table}


\subsection{Daily Change}

To test whether cumulative weather exposure influenced day-to-day roost dynamics, we analyzed 96 consecutive-day pairs from the same deployment dataset using biologically-aligned temporal windows. The primary analysis employed a sunset window spanning from the previous day's maximum count to the current day's last observation (mean duration = 29.6 hours), capturing the full period from peak aggregation through the subsequent roosting decision. A secondary analysis using fixed 24-hour windows (n = 94 pairs) provided a sensitivity test of our findings.

Daily maximum cluster sizes ranged from 0 to 770 butterflies (mean = 134.7 ± 138.1), with day-to-day changes in maximum count ranging from losses of 376 butterflies to gains of 464 butterflies (mean change = -10.5 ± 111.6). Within the sunset windows, maximum wind gusts ranged from 2.0 to 12.8 m/s (mean = 4.5 ± 1.8 m/s), with all observation windows exceeding the proposed 2 m/s threshold. Cumulative direct sun exposure varied from 0 to 1,122 butterfly-observations in sunlight per window (mean = 139.8 ± 206.9), reflecting diverse thermal exposure conditions across monitoring days.
