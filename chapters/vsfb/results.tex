\section{Results}

\subsection{Descriptive Statistics}

Environmental conditions varied substantially across the 78-day monitoring period at Spring Canyon and UDMH during the 2023-2024 overwintering season. The dataset comprised 1,894 observations collected at 30-minute intervals during daylight hours (07:00--17:00) from November 17, 2023, to February 4, 2024, totaling 947 observation hours across 115 unique deployment-day combinations.

Wind speeds ranged from complete calm to moderately strong conditions, with maximum gusts reaching 12.4 m/s (mean = 2.2 ± 1.4 m/s, median = 2.2 m/s). The interquartile range of 1.3--3.0 m/s indicated that most observations occurred under relatively mild wind conditions. Temperature showed considerable variation throughout the monitoring period, ranging from 3.0 to 30.0°C (mean = 14.6 ± 3.8°C, median = 14.0°C), with an interquartile range of 12.5--17.0°C typical of California coastal winter conditions. Direct solar exposure occurred in 31.7\% of observations (n = 601), with butterflies actively basking when present in sunlight, averaging 17.0 individuals in direct sun (range: 1--295).

Monarch abundance exhibited high variability across sites and time periods. Butterfly counts ranged from 0 to 770 individuals per observation, with a mean of 81.4 ± 100.0 butterflies and a median of 37 butterflies. The wide interquartile range (9--119 butterflies) reflected substantial variation in cluster sizes. Zero-count observations, representing either the beginning of cluster formation or cluster dissolution, comprised 2.3\% of the dataset (n = 43).

Cluster sizes varied markedly among the 10 deployment locations. Site SC10 recorded the largest aggregation with 770 monarchs, while mean abundances ranged from 0 at SC9 to 325.8 at UDMH2. Eight deployments observed maximum cluster sizes exceeding 100 butterflies, with mean maximum cluster size across sites reaching 315.6 individuals. This variation in cluster sizes across deployments reflects the heterogeneous distribution of monarchs across overwintering microhabitats within the study sites.

The comprehensive temporal coverage, with observations at 30-minute intervals capturing 16.5 observations per deployment-day on average, provided fine-scale resolution of monarch behavioral responses to changing environmental conditions. Peak observation activity occurred at 16:00 hours (196 observations), corresponding with afternoon warming periods when monarchs typically exhibit increased movement.

\subsection{Responsive Change}

% A few introductory sentences here to remind the reader of what this analysis is, and some context to help them understand what comes next. "Responsive Change" refers to my 30 minute analysis. All figures and information should come from supplemental/results/30_min. 

\subsubsection{Model Selection}

% Here you can briefly explain how many models were tested. The AIC table should go in this section and be the bulk of the content. 

\subsubsection{Analysis of Best Fit Model}

% Here I want you to idnetify the best fit model and include the formula for the model written in plain english. Give a bit of flavor about the description of the model, then provide a summary of the model results as a table. We should also include the partial effects plot here with a description (below), and the interaction plot separately, again with a description. OK to leave the description to me fill in for the interaction. I'm going to paste the old description I have for the partial effects below, but make sure the numbers are correct for the new analysis. 

% "Partial effects for the three smooth main effects are shown in Figure 1.1. The previous butterfly count showed a significant non‑linear negative relationship (p < 0.001) consistent with proportionally greater departures from larger aggregations. Time since sunrise captured a significant diurnal pattern (p < 0.001) with morning departures and afternoon returns. Temperature showed a marginally non-significant trend (p = 0.057) suggesting possible effects below the 12.7–16°C flight threshold."

\subsubsection{Model Diagnostics}

% We need to show the combined residuales vs fitted vales and qq plot here. I also want to show the diag_acf plot as well. Take a try at describing them in text. Only for the 30_min analysis. 

\subsubsection{Sensitivity Analysis}

% Note here that we basically redid the whole analysis, but instead of using max gust, we used minutes above 2 m/s to test the threshold hypothesis. The best model followed the same structure as above, but had a lower R2 value and was not the decisive winner in the AIC comparison, sharing model weight with the next best one. For this reason, we are preferring the direct measure of wind instead of the threshold. Still, you should include the AIC table, and I want to include the interaction plot as well to lend evidence that wind is not an important factor. The other plots can be left out. You can find what you need for this section here: supplemental/results/30_min_threshold. 

\subsection{Daily Change}

