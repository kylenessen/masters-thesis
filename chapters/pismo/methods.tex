\section{Materials and Methods}
\subsection{Study Site}
Pismo State Beach Monarch Butterfly Grove (hereafter "Pismo") is located in San Luis Obispo County, California (35.12940° N, 120.628° W). The site encompasses approximately 5 hectares (12.4 acres) and is characterized by a mature grove of blue gum eucalyptus (\textit{Eucalyptus globulus}). The grove is situated approximately 0.5 km from the Pacific Ocean, which lies directly to the west. Between the Pacific Ocean and Pismo is the North Beach Campground and Pismo Beach Golf Course. Both facilities are minimimally developed with few buildings, scattered trees, and largely open grassy areas. North and East of the Pismo is primarily medium density residential neighborhoods. The houses do not exceed two stories. TO the south is Oceano Dunes Natural Preserve State Park, which is characterized by mature coastal dune scrub habitat, which is primarily defined by low rolling dunes with low perennial vegetation. Overall, the Eucaylptus trees found within Pismo are the dominant feature in the local area. the  It is managed by California State Parks. 
-
% Discuss other trees, heights of trees, and surrounding features. 
% Include figure of the surounding area with a digital surface model. Highlight Pismo Grove with a boundary and show that the areas around the site are relatively flat. Maybe contours? 

Pismo was selected as the primary study site for several key characteristics. The site consistently supports one of the largest aggregations of overwintering monarch butterflies (\textit{Danaus plexippus}) in California, routinely ranking among the top ten overwintering sites by population size \autocite{westernmonarchcount2023}. Even during years of low monarch abundance, such as 2024, Pismo maintains a presence of butterflies while many other sites remain vacant.

The site's physical characteristics make it particularly suitable for wind analysis. The western exposure to the Pacific Ocean provides an unobstructed wind corridor, minimizing confounding topographical effects. The surrounding terrain is predominantly flat, and nearby anthropogenic structures do not exceed two stories in height, representing less than 20\% of the canopy height of the grove's mature eucalyptus trees.

Additionally, Pismo's extensive history of monarch butterfly research and consistent population monitoring provides valuable historical context for this study. The site's well-documented population counts, conducted at regular intervals, offer opportunities for correlating wind patterns with butterfly abundance and distribution patterns.

\subsection{Study Site}
